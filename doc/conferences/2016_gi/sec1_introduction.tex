% "\firstsection{Introduction}" is in "main.tex"

Exascale computers, by definition capable of performing at least one
exaflop ($10^{18}$ floating point operations per second), are
anticipated by 2018~\cite{kogge2013exascale}.
This will be several orders of magnitude greater than the fastest computers
currently available. Expectations will be high that software running on these computers will scale similarly.
Experience has shown, however, that such improvements will not be
achievable without similarly dramatic innovations in software,
particularly programming models and runtime design.

Until recent years, application performance has increased in
correspondence with Moore's law (which is not so much a law as an
observation): the number of transistors within an integrated circuit
doubled approximately every two years. As we reached a limit on the
number of transistors a single core could contain, hardware architects
had to look for other ways to keep up with performance advancement
expectations. The most common solution today is multiple cores per
chip. In order to take advantage of these hardware advances,
applications often require redesign.

In addition to the changing software architectures, the amount of
output data expected from high-performance computing (HPC)
applications also scales with compute power: larger computers are
expected to manage larger amounts of data. An additional expectation
is that of ``real time'', dynamic display of data.

These demands have produced a need for rendering and visualization
systems which can take advantage of distributed systems. These systems
may be stand-alone or they may add functionality to existing HPC
applications. This paper proposes one such system for ray tracing, a
commonly-used rendering technique, using the Intel Concurrent
Collections (CnC) programming model. Although CnC runs on current
distrbuted hardware, it is ultimately intended for use on exascale
computers.


%%% Local Variables: 
%%% mode: latex
%%% TeX-master: "main"
%%% End: 
