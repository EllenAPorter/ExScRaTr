% $Id: template.tex 11 2007-04-03 22:25:53Z jpeltier $

\documentclass{vgtc}                          % final (conference style)
%\documentclass[review]{vgtc}                 % review
%\documentclass[widereview]{vgtc}          % wide-spaced review
%\documentclass[preprint]{vgtc}               % preprint
%\documentclass[electronic]{vgtc}             % electronic version

%% Uncomment one of the lines above depending on where your paper is
%% in the conference process. ``review'' and ``widereview'' are for review
%% submission, ``preprint'' is for pre-publication, and the final version
%% doesn't use a specific qualifier. Further, ``electronic'' includes
%% hyperreferences for more convenient online viewing.

%% Please use one of the ``review'' options in combination with the
%% assigned online id (see below) ONLY if your paper uses a double blind
%% review process. Some conferences, like IEEE Vis and InfoVis, have NOT
%% in the past.

%% Figures should be in CMYK or Grey scale format, otherwise, colour 
%% shifting may occur during the printing process.

%% These three lines bring in essential packages: ``mathptmx'' for Type 1 
%% typefaces, ``graphicx'' for inclusion of EPS figures. and ``times''
%% for proper handling of the times font family.

\usepackage{mathptmx}
\usepackage{graphicx}
\usepackage{times}
\usepackage{cite} 
\usepackage{subcaption}

%% We encourage the use of mathptmx for consistent usage of times font
%% throughout the proceedings. However, if you encounter conflicts
%% with other math-related packages, you may want to disable it.

%% If you are submitting a paper to a conference for review with a double
%% blind reviewing process, please replace the value ``0'' below with your
%% OnlineID. Otherwise, you may safely leave it at ``0''.
\onlineid{0}

%% declare the category of your paper, only shown in review mode
\vgtccategory{Research}

%% allow for this line if you want the electronic option to work properly
\vgtcinsertpkg

%% In preprint mode you may define your own headline.
%\preprinttext{To appear in an IEEE VGTC sponsored conference.}

%% Paper title.

\title{An Prospective Approach to Ray Tracing on Exascale Computers}

%% This is how authors are specified in the conference style

%% Author and Affiliation (single author).
%%\author{Roy G. Biv\thanks{e-mail: roy.g.biv@aol.com}}
%%\affiliation{\scriptsize Allied Widgets Research}

%% Author and Affiliation (multiple authors with single affiliations).
\author{
  Ellen A. Porter
  \thanks{e-mail: \texttt{ellen.porter@wsu.edu}} %
  \and
  Robert R. Lewis
  \thanks{e-mail: \texttt{bobl@tricity.wsu.edu}}}

\affiliation{
  \scriptsize Washington State University, Tri-Cities \\
  Program in Engineering and Computer Science}

%% Author and Affiliation (multiple authors with multiple affiliations)
%%\author{Roy G. Biv\thanks{e-mail: roy.g.biv@aol.com}\\ %
%%        \scriptsize Starbucks Research %
%%\and Ed Grimley\thanks{e-mail:ed.grimley@aol.com}\\ %
%%     \scriptsize Grimley Widgets, Inc. %
%%\and Martha Stewart\thanks{e-mail:martha.stewart@marthastewart.com}\\ %
%%     \parbox{1.4in}{\scriptsize \centering Martha Stewart Enterprises \\ Microsoft Research}}

%% A teaser figure can be included as follows, but is not recommended since
%% the space is now taken up by a full width abstract.
%\teaser{
%  \includegraphics[width=1.5in]{sample.eps}
%  \caption{Lookit! Lookit!}
%}

\abstract{ % 
  Exascale computers, defined as being capable of performing at least %
  one exaflop ($10^{18}$ floating point operations per second) are
  anticipated to emerge in the next several years. Reaching this scale
  of computation will require significant hardware and software
  changes for high-performance computing (HPC). In this paper we
  explore those changes and the impact they will have on programming
  models and application design. We then look at one specific model,
  the Concurrent Collections Programming Model (CnC) and explore how
  we can use CnC and Intel's Embree Ray Tracing Engine to build a
  scalable ray tracing system ready for exascale.
}

%% ACM Computing Classification System (CCS). 
%% See <http://www.acm.org/class/1998/> for details.
%% The ``\CCScat'' command takes four arguments.

\CCScatlist{ 
\CCScat{}{Computing methodologies}{Ray tracing}{}
\CCScat{}{Computing methodologies}{Parallel programming languages}{}
\CCScat{}{Hardware}{Emerging architectures}{}
}

%% Copyright space is enabled by default as required by guidelines.
%% It is disabled by the 'review' option or via the following command:
% \nocopyrightspace

%%%%%%%%%%%%%%%%%%%%%%%%%%%%%%%%%%%%%%%%%%%%%%%%%%%%%%%%%%%%%%%%
%%%%%%%%%%%%%%%%%%%%%% START OF THE PAPER %%%%%%%%%%%%%%%%%%%%%%
%%%%%%%%%%%%%%%%%%%%%%%%%%%%%%%%%%%%%%%%%%%%%%%%%%%%%%%%%%%%%%%%%

\begin{document}

%% The ``\maketitle'' command must be the first command after the
%% ``\begin{document}'' command. It prepares and prints the title block.

%% the only exception to this rule is the \firstsection command

\firstsection{Introduction}

\maketitle

%% \section{Introduction} 

Exascale computers, by definition capable of performing at least one
exaflop ($10^{18}$ floating point operations per second), are
anticipated by 2018.
% RRL: add citation
This will be several orders of magnitude greater than the fastest computers
currently available. Expectations will be high that software running on these computers will scale similarly.
Experience has shown, however, that such improvements will not be
achievable without similarly dramatic innovations in software,
particularly programming models and runtime design.

Until recent years, application performance has increased in
correspondence with Moore's law (which is not so much a law as an
observation): the number of transistors within an integrated circuit
doubled approximately every two years. As we reached a limit on the
number of transistors a single core could contain, hardware architects
had to look for other ways to keep up with performance advancement
expectations. The most common solution today is multiple cores per
chip. In order to take advantage of these hardware advances,
applications often require redesign.

In addition to the changing software architectures, the amount of
output data expected from high-performance computing (HPC)
applications also scales with compute power: larger computers are
expected to manage larger amounts of data. An additional expectation
is that of ``real time'', dynamic display of data.

These demands have produced a need for rendering and visualization
systems which can take advantage of distributed systems. These systems
may be stand-alone or they may add functionality to existing HPC
applications. This paper proposes one such system for ray tracing, a
commonly-used rendering technique, using the Intel Concurrent
Collections (CnC) programming model. Although CnC runs on current
distrbuted hardware, it is ultimately intended for use on exascale
computers.

\section{Reaching Exascale}

For the past two decades high performance computing (HPC) progression
has been driven by Moore's law. Until 2004, performance of single-core
microprocessors increased as predicted as a result of smaller and
faster transistors being developed. In 2004, this advancement trend
shifted as we reached an inflection point caused by a chip’s power
dissipation ~\cite{kogge2013exascale}. Unable to sufficiently and
inexpensively cool a chip, chip designers looked for other ways to
increase performance. This came in the form of multi-core processors
which are now the building blocks of many HPC systems.

The introduction of multi-core processors on a single node of a
cluster caused a shift in parallel application design. Programs using
the standard Message Passing Interface (MPI) library
~\cite{Snir:1998:MCR:552013} could not exploit parallelism on a single
node without a rewrite of the underlying algorithms. This resulted in
the emergence of hybrid systems that mix MPI and the Open
Multi-Processing (OpenMP) ~\cite{openmp08} libraries. OpenMP being
designed for shared memory multiprocessors, each node would execute an
OpenMP program controlled overall by MPI using a fixed number of
threads to execute a single work-sharing construct such as a parallel
loop ~\cite{gropp2013programming}.

As we look towards the next generation of HPC systems a shift in
application design will once again be necessary to reach exascale
performance. On-chip parallelism along with reduced data movement will
be critical for an applications success. Unfortunately, conventional
language semantics will not be sufficient to exploit the architecture
advances being developed such as inter-core message queues. Therefore,
new high-performance parallel programming models and smarter runtimes
are being developed.

The majority of these models are data-centric rather than
compute-centric, which allows the runtime scheduler to prioritize
scheduling computation on nodes or cores where the required data
already resides rather than the next available processor
~\cite{kogge2013exascale}. This kind of model will reduce
communication which is the predicted bottle neck for exascale systems.

\section{The CnC Programming Model}

Concurrent Collections (CnC) is one such data-centric programming
model, the deterministic semantics of which allow a task-based runtime
to programmatically exploit parallelism. In this model, algorithms are
designed based on their data and control dependencies
~\cite{budimlicconcurrent}. The specifics regarding the execution of
the algorithm is then abstracted out of the implementation. This
allows the runtime to optimally decide when and where to schedule
computation. Hints can also be provided to the runtime through a
separate file called a ``tuning specification''.

The CnC model is built on three key constructs: step collections, data collections, and control collections ~\cite{budimlicconcurrent}.  A step collection defines computation, an instance of which consumes and produces data.  The consumed and produced data, or data items, belong to data collections.  Data items within a data collection are indexed using item tags.
%% RRL: We need to say more about tags.
Finally the control collection describes the prescription, or creation, of step instances.  The relationship between these three collections is defined statically in an input file called a ``CnC graph''.

Developing a CnC application begins with designing the CnC graph file.  An algorithm is broken down into computation steps, instances of which correspond to different input arguments.  These steps along with the data collections become nodes in a graph.  Each step can optionally consume data, produce data, and prescribe additional computation.  These relationships, producer, consumer, and control, define the edges in the graph and will dynamically be satisfied as the program executes.

\begin{figure*}[!htb]
  \centering
  \includegraphics[width=\textwidth]{drawings/CnC.pdf}
  \caption{CnC Graph.}
  \label{fig:cnc}
\end{figure*}
%% RRL: We need to say more about this diagram.

The next and final required step in producing a CnC application is to implement the step logic.  The flow within a single CnC step is as follows: consume, compute, and produce.  This ordering is required as there is no guarantee the data a step needs will be ready when the step beings executing.  Internally, CnC will attempt to retrieve the data, if it is not ready, the step will halt execution and try again later.  To improve performance, hints can be provided through the tuning specification to ensure steps are only prescribed and scheduled for execution when their required input data is ready.
%% RRL: It seems odd that so necessary a consideration that a step
%% needs to have its data ready before it starts running is considered
%% ``tuning''.

\section{Limitations of Current Ray Tracing Algorithms}

Traditional ray tracing algorithms are embarrassingly parallel as no
primary ray depends on any other ray. The data needed by each
individual ray, however, varies widely as its path is traced,
especially as regards secondary or other rays. Acceleration
structures, such as k-d trees have been developed to increase ray
tracing performance. As the data scales up however, it is no longer
possible to store an entire data set in an acceleration structure in
shared memory.

One solution is to implement data decomposition. Each node on a
distributed system is then responsible for a subset of the domain.
Primary rays and secondary rays are then communicated across nodes as
the algorithm executes. These types of models typically rely on
expensive preprocessing steps that help to balance both the data
distribution and rendering work evenly across nodes
~\cite{navratil2014dynamic}.

\section{CnC Ray Tracing Implementation}

Implementing ray tracing using CnC allows the algorithm to run on a
distributed system which is planned to be extensible to exascale
computers and reduces the need for expensive preprocessing seen with
many current systems. This is due to the dynamic nature of CnC
execution and is similar to the algorithm proposed by Navratil et al.
~\cite{navratil2014dynamic}. The CnC implementation beings by
partitioning data into voxels, this data is then distributed
dynamically at runtime.
%% RRL: Aren't partitioning and distibuting the same thing? You seem
%% to be saying that the data is partitioned both before and during
%% runtime.
The ray tracing portion of the algorithm is
iterative. Primary rays are sent into the system from the camera and
may give rise to secondary rays (shadow, reflection, refraction, etc.)
until all rays are fully traced.
%% RRL: Not clear what we mean by ``fully traced''.

Figure~\ref{fig:cnc} shows the proposed CnC graph for distributed ray
tracing.
%% What does the color coding mean?
It represents data collections as ovals and step collections
as rectangles, each with a collection tag
%% RRL: ``collection tag'' vs. ``tag collection(s)''?
(in upper case on its first line) and associated item tag(s) (in
square brackets on its subsequent lines). It also shows the
dependencies between them. The control collection for the proposed
model is static for all steps except TRACE\_VOXEL
%% Why?
and defined in an initialization step. The graph begins
execution when the object data, light data, and camera data are
provided, and terminates when the image is produced.
%% RRL: What if there are multiple frames?

\subsection{Tag Collections}

The tag collections are different for most data and step collections
%% RRL: Are these collections of ``item tags''
but share common elements. FRAME refers to one specific frame in the
case of an animation. INSTANCE refers to the current iteration.
%% RRL: What about this algorithm is iterative?
I, J, K are iterators over spatial data, or in the case of LIGHT, the
light index.
%% RRL: It's confusing to use ``I'' twice like this.

\subsection{Data Collections}

%% RRL: It would be more consistent to either drop the ``_DATA''
%% suffix or use it consistently.

OBJECT\_DATA contains input data for the scene from the environment.
In our case, it is in the form of a Wavefront ``\texttt{.obj}'' file.
DECOMPOSE\_DOMAIN splits this data into voxels and produces
VOXEL\_OBJECT\_DATA. Triangles that span multiple voxels are
duplicated. The LIGHT data collection contains data pertaining to any
light sources. VOXEL\_LIGHT\_DATA contains the same information as
LIGHT plus a traced light mesh for each wall of a voxel. CAMERA
contains the location and direction of the camera. RAY\_PACKET
contains all the rays that intersect a voxel wall for a given wall and
iteration. IMAGE contains the final image data.

\subsection{Step Collections}

\subsubsection{Decompose Domain}

DECOMPOSE\_DOMAIN takes the data to be traced as input and produces
subsets of that data based on voxel decomposition. As load balancing
is not a concern
% RRL: Have you adequately explained why not?
a fast uniformly spaced geometric distribution is sufficient. The
number of voxels produced is set at runtime and should be more than
the number of nodes available.

\subsubsection{Distribute Lights}

In order to reduce communication of secondary rays, DISTRIBUTE\_LIGHTS
lights is responsible for distributing light information to each
voxel. This guarantees each voxel will only need to communicate with
its direct neighbors. The light information produced for each voxel
contains the original light sources as well as a light source mesh for
each light and each wall of the voxel. The mesh is produced by tracing
rays from each light source to a uniform grid on a voxel wall. Where the
rays intersect the wall, a new directional light source is created to
light the cell. If the ray is blocked, the point light source will be
tagged as in shadow.

\begin{figure}[!htb]
\centering
\begin{subfigure}{.49\columnwidth}
 \centering
  \includegraphics[width=.98\columnwidth]{drawings/Lights1.pdf}
  \caption{Initial light rays}
\end{subfigure}
\begin{subfigure}{.49\columnwidth}
 \centering
  \includegraphics[width=.98\columnwidth]{drawings/Lights2.pdf}
  \caption{Point light sources}
\end{subfigure}
\caption{Light Ray Distribution}
\label{fig:light}
\end{figure}

\subsubsection{Distribute Rays}

DISTRIBUTE\_RAYS is responsible for sending each voxel its first iteration of ray information.  This is an empty set of data for all voxels except the voxel containing the camera if the camera is positioned within the domain.  If the camera is outside the domain, multiple voxels may receive data.

\subsubsection{Trace Voxel}

TRACE\_VOXEL is the heart of the application.  This step is iterative and prescribes the next iteration as long as there are rays still to trace and it has not reached a maximum threshold.  Trace voxel consumes ray packets from each of its neighbors.  It then traces the rays over its subset of the domain.  If a ray intersects with an object, secondary rays from each light source are considered if the corresponding point light source from the voxels light mesh is not in shadow.  Rays that reach the voxel walls are collected and passed to the corresponding neighbor on the next iteration.  As each CnC step instance will eventually be executed on a single node of a cluster the step code is integrating with Embree, Intel’s ray tracing kernel, in order to optimize per node performance.

\begin{figure}[!htb]
\centering
\begin{subfigure}{.49\columnwidth}
 \centering
  \includegraphics[width=.98\columnwidth]{drawings/Trace1.pdf}
  \caption{Distribute rays}
\end{subfigure}
\begin{subfigure}{.49\columnwidth}
 \centering
  \includegraphics[width=.98\columnwidth]{drawings/Trace2.pdf}
  \caption{Trace voxel; iteration 1}
\end{subfigure}
\begin{subfigure}{.49\columnwidth}
 \centering
  \includegraphics[width=.98\columnwidth]{drawings/Trace3.pdf}
  \caption{Trace voxel; iteration 2}
\end{subfigure}
\begin{subfigure}{.49\columnwidth}
 \centering
  \includegraphics[width=.98\columnwidth]{drawings/Trace4.pdf}
  \caption{Trace voxel; iteration 3}
\end{subfigure}
\caption{Iterative Trace Voxel}
\label{fig:trace}
\end{figure}

\subsubsection{Produce Image}
When TRACE\_VOXEL has converged, a final set of ray packets will be
produced. Each ray in that packet contains the information necessary
to produce the final image; merging these is the responsibility of the
PRODUCE\_IMAGE step.

\section{Future Directions}

\subsection{Hierarchy}

\subsection{Dynamic scenes}

\subsection{checkpoint and restart} 

\subsection{Global illumination}

\section{Conclusion}

%% if specified like this the section will be ommitted in review mode
\acknowledgements{
The authors wish to thank A, B, C.  CnC team at RICE? Intel?}

\bibliographystyle{abbrv}

%%use following if all content of bibtex file should be shown
%\nocite{*}
\bibliography{bibliography}{}

\end{document}
