\section{Reaching Exascale}

For the past two decades high performance computing (HPC) progression
has been driven by Moore's law. Until 2004, performance of single-core
microprocessors increased as predicted as a result of smaller and
faster transistors being developed. In 2004, this advancement trend
shifted as we reached an inflection point caused by a chip’s power
dissipation ~\cite{kogge2013exascale}. Unable to sufficiently and
inexpensively cool a chip, chip designers looked for other ways to
increase performance. This came in the form of multi-core processors
which are now the building blocks of many HPC systems.

The introduction of multi-core processors on a single node of a
cluster caused a shift in parallel application design. Programs using
the standard Message Passing Interface (MPI) library
~\cite{Snir:1998:MCR:552013} could not exploit parallelism on a single
node without a rewrite of the underlying algorithms. This resulted in
the emergence of hybrid systems that mix MPI and the Open
Multi-Processing (OpenMP) ~\cite{openmp08} libraries. OpenMP being
designed for shared memory multiprocessors, each node would execute an
OpenMP program controlled overall by MPI using a fixed number of
threads to execute a single work-sharing construct such as a parallel
loop ~\cite{gropp2013programming}.

As we look towards the next generation of HPC systems a shift in
application design will once again be necessary to reach exascale
performance. On-chip parallelism along with reduced data movement will
be critical for an applications success. Unfortunately, conventional
language semantics will not be sufficient to exploit the architecture
advances being developed such as inter-core message queues. Therefore,
new high-performance parallel programming models and smarter runtimes
are being developed.

The majority of these models are data-centric rather than
compute-centric, which allows the runtime scheduler to prioritize
scheduling computation on nodes or cores where the required data
already resides rather than the next available processor
~\cite{kogge2013exascale}. This kind of model will reduce
communication which is the predicted bottle neck for exascale systems.


%%% Local Variables: 
%%% mode: latex
%%% TeX-master: "main"
%%% End: 
