\section{Limitations of Current Ray Tracing Algorithms}
\label{sec:limitations}

Traditional parallel ray tracing algorithms are ``embarrassingly'' so
as no ray depends on any other ray, at least on the same level of the
ray tracing tree. The data needed by each individual ray, however,
varies widely as its path is traced. Acceleration structures such as
k-d trees
% RRL: citation, also mention SEADS and octrees: see Pharr & Humphries
have been developed to increase ray tracing performance. As the size
of the scene increases, however, it is no longer possible to store an
entire data set in an acceleration structure in shared memory.

One solution is to implement data decomposition. Each node on a
distributed system is then responsible for a subset of the domain.
Primary, secondary, and subsequent rays are then communicated across
nodes as the algorithm executes. These types of models typically rely
on expensive preprocessing steps that help to balance both the data
distribution and rendering work evenly across nodes
~\cite{navratil2014dynamic}.

Load balancing, a significant bottleneck on today’s systems, may not
be easily implemented on exascale systems. The proposed smarter
programming models and runtimes, on the other hand, will allow for
scheduling and data movement decisions to be made at runtime which
will help reduce imbalance in a system. Data and computation can be
dynamically migrated off of overworked nodes (assuming a properly
sized granularity for tasks and data).

%%% Local Variables: 
%%% mode: latex
%%% TeX-master: "main"
%%% End: 
