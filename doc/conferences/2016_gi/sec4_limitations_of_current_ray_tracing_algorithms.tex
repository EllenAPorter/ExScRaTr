\section{Limitations of Current Ray Tracing Algorithms}

% RRL: I will add "lmprop" of ray tracer here.

Traditional ray tracing algorithms are embarrassingly parallel as no
primary ray depends on any other ray. The data needed by each
individual ray, however, varies widely as its path is traced,
especially as regards secondary or other rays. Acceleration
structures, such as k-d trees have been developed to increase ray
tracing performance. As the data scales up however, it is no longer
possible to store an entire data set in an acceleration structure in
shared memory.

One solution is to implement data decomposition. Each node on a
distributed system is then responsible for a subset of the domain.
Primary rays and secondary rays are then communicated across nodes as
the algorithm executes. These types of models typically rely on
expensive preprocessing steps that help to balance both the data
distribution and rendering work evenly across nodes
~\cite{navratil2014dynamic}.

%%% Local Variables: 
%%% mode: latex
%%% TeX-master: "main"
%%% End: 
