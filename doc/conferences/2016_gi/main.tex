% $Id: template.tex 11 2007-04-03 22:25:53Z jpeltier $

\documentclass{vgtc}                          % final (conference style)
%\documentclass[review]{vgtc}                 % review
%\documentclass[widereview]{vgtc}          % wide-spaced review
%\documentclass[preprint]{vgtc}               % preprint
%\documentclass[electronic]{vgtc}             % electronic version

%% Uncomment one of the lines above depending on where your paper is
%% in the conference process. ``review'' and ``widereview'' are for review
%% submission, ``preprint'' is for pre-publication, and the final version
%% doesn't use a specific qualifier. Further, ``electronic'' includes
%% hyperreferences for more convenient online viewing.

%% Please use one of the ``review'' options in combination with the
%% assigned online id (see below) ONLY if your paper uses a double blind
%% review process. Some conferences, like IEEE Vis and InfoVis, have NOT
%% in the past.

%% Figures should be in CMYK or Grey scale format, otherwise, colour 
%% shifting may occur during the printing process.

%% These three lines bring in essential packages: ``mathptmx'' for Type 1 
%% typefaces, ``graphicx'' for inclusion of EPS figures. and ``times''
%% for proper handling of the times font family.

\pagestyle{plain}

\usepackage{mathptmx}
\usepackage{graphicx}
\usepackage{times}
\usepackage{cite} 
\usepackage{subcaption}
\usepackage{listings}

%% We encourage the use of mathptmx for consistent usage of times font
%% throughout the proceedings. However, if you encounter conflicts
%% with other math-related packages, you may want to disable it.

%% If you are submitting a paper to a conference for review with a double
%% blind reviewing process, please replace the value ``0'' below with your
%% OnlineID. Otherwise, you may safely leave it at ``0''.
\onlineid{0}

%% declare the category of your paper, only shown in review mode
\vgtccategory{Research}

%% allow for this line if you want the electronic option to work properly
\vgtcinsertpkg

%% In preprint mode you may define your own headline.
%\preprinttext{To appear in an IEEE VGTC sponsored conference.}

%% Paper title.

\title{Anticipating Ray Tracing on Exascale Computers}

%% This is how authors are specified in the conference style

%% Author and Affiliation (single author).
%%\author{Roy G. Biv\thanks{e-mail: roy.g.biv@aol.com}}
%%\affiliation{\scriptsize Allied Widgets Research}

%% Author and Affiliation (multiple authors with single affiliations).
\author{
  Ellen A. Porter
  \thanks{e-mail: \texttt{ellen.porter@wsu.edu}} %
  \and
  Robert R. Lewis
  \thanks{e-mail: \texttt{bobl@tricity.wsu.edu}}}

\affiliation{
  \scriptsize Washington State University, Tri-Cities \\
  Program in Engineering and Computer Science}

%% Author and Affiliation (multiple authors with multiple affiliations)
%%\author{Roy G. Biv\thanks{e-mail: roy.g.biv@aol.com}\\ %
%%        \scriptsize Starbucks Research %
%%\and Ed Grimley\thanks{e-mail:ed.grimley@aol.com}\\ %
%%     \scriptsize Grimley Widgets, Inc. %
%%\and Martha Stewart\thanks{e-mail:martha.stewart@marthastewart.com}\\ %
%%     \parbox{1.4in}{\scriptsize \centering Martha Stewart Enterprises \\ Microsoft Research}}

%% A teaser figure can be included as follows, but is not recommended since
%% the space is now taken up by a full width abstract.
%\teaser{
%  \includegraphics[width=1.5in]{sample.eps}
%  \caption{Lookit! Lookit!}
%}

\abstract{ %
  Exascale computers, defined as being capable of performing at least
  one exaflop ($10^{18}$ floating point operations per second) are
  anticipated to emerge in the next several years. Reaching this scale
  of computation will require hardware and software changes for
  high-performance computing (HPC). Applications will need to adapt as
  the architecture of supercomputers change. Some studies suggest
  ``communication avoiding'' algorithms might be the most performant
  design for future systems. This has created an interest in the
  development of scalable visualization algorithms and techniques.

  In this paper we explore the impact exascale hardware will have on
  programming models and application design. We then look at one
  specific model, Concurrent Collections (CnC), and explore how we can
  use it and Intel's Embree Ray Tracing Engine to build a scalable ray
  tracing system with an emphasis on communication avoidance and
  extension for exascale. As exascale hardware does not yet exist,
  this work is speculative, but we hope it serves as a foundation for
  future discussion and work. %
}

%% ACM Computing Classification System (CCS). 
%% See <http://www.acm.org/class/1998/> for details.
%% The ``\CCScat'' command takes four arguments.

\CCScatlist{ 
\CCScat{}{Computing methodologies}{Ray tracing}{}
\CCScat{}{Computing methodologies}{Parallel programming languages}{}
\CCScat{}{Hardware}{Emerging architectures}{}
}

%% Copyright space is enabled by default as required by guidelines.
%% It is disabled by the 'review' option or via the following command:
% \nocopyrightspace

%%%%%%%%%%%%%%%%%%%%%%%%%%%%%%%%%%%%%%%%%%%%%%%%%%%%%%%%%%%%%%%%
%%%%%%%%%%%%%%%%%%%%%% START OF THE PAPER %%%%%%%%%%%%%%%%%%%%%%
%%%%%%%%%%%%%%%%%%%%%%%%%%%%%%%%%%%%%%%%%%%%%%%%%%%%%%%%%%%%%%%%%

\begin{document}

%% The ``\maketitle'' command must be the first command after the
%% ``\begin{document}'' command. It prepares and prints the title block.

%% the only exception to this rule is the \firstsection command

\firstsection{Introduction}
\label{sec:intro}

\maketitle

%% \section{Introduction} 

% "\firstsection{Introduction}" is in "main.tex"

Achieving the performance expected from an exascale computer will
require modifications to current hardware architecture which will in
turn affect programming models and runtime\footnote{ %
  We use the term ``runtime'' in the sense of a library or libraries
  compiled into and running as part of anapplication which is not
  specific to the application but which moderates its interface (e.g.
  memory management, thread prioritization, etc.) with the operating
  system. It's not just a ``library'', as it may have its own threads
  or other execution units. %
} design. Until recent years, performance increased in keeping with
Moore's ``Law'' (which is really more of an observation): The number
of transistors within an integrated circuit doubled approximately
every two years. As we reached a limit on the number of transistors a
single chip could contain, hardware architects had to look for other
ways to keep up with performance advancement expectations. In most
cases, this involved a greater emphasis on parallelism. Consequently,
in order to take advantage of hardware advances, applications,
runtimes, and programming models have often required redesign, if not
reimplementation.

As we look towards the next generation of high-performance computing
(HPC) systems, a shift in application design is again anticipated,
this time to reach exascale performance. On-chip parallelism along
with reduced data movement will be critical for applications to make
optimal use of the hardware and minimize power consumption.

Unfortunately, conventional language semantics will not be sufficient
to exploit the architectural advances being developed such as
inter-core message queues. Therefore, new parallel programming models
and smarter runtimes are being designed. The majority of these models
are ``data-centric'' rather than ``compute-centric'': They allow, for
instance, the runtime scheduler to prioritize scheduling computation
on nodes or cores where the required data already resides rather than
% RRL: Can we standardize on (flaxible) OpenCL nomenclature for
% parallelism?
the next available processor ~\cite{kogge2013exascale}. This kind of
model will reduce communication which is the predicted bottle neck for
exascale systems.

The data produced as output from HPC applications such as fluid
simulations or finite-element models tends to scale in size with
compute power. This is expected to occur with exascale systems as well
and has produced a need for visualization algorithms that can take
advantage of distributed systems as well as an opportunity to design
algorithms that can be integrated into HPC applications to produce
results during execution. Section~\ref{sec:raytracing} proposes one
such design for ray tracing, a commonly used rendering technique,
using the Intel Concurrent Collections (CnC) programming model.

The rest of this paper is organized as follows: We start with a description of exascale along with a description of the projected trends in programming models that will perform well on exascale.  We then explore one programming model, CnC, that is expected to map well to exascale systems.  After describing the CnC programming model we analyze current ray tracing algorithms and propose places for improvement for exascale.  Specifically, we look at ways we can reduce communication overhead within the algorithm.  We then describe the implementation details of a ray tracer developed in CnC and look at how it might perform on future exascale hardware.  Finally we conclude with a section on future work.

The work we present here is an extension of that in \cite{porter2014cnc}.

%%% Local Variables: 
%%% mode: latex
%%% TeX-master: "main"
%%% End: 


\section{Reaching Exascale}

For the past two decades high performance computing (HPC) progression has been driven by Moore's law which states microprocessor performance and memory chip density increase exponentially over time. Until 2004, performance of single-core microprocessors increased as predicted as a result of smaller and faster transistors being developed. In 2004 this advancement trend shifted as we reached an inflection point caused by a chip’s power dissipation ~\cite{kogge2013exascale}. Unable to sufficiently and inexpensively cool a chip, chip designers looked for other ways to increase performance. This came in the form of multi-core processors which are now the building blocks of many HPC systems.

The introduction of multi-core processors on a single node of a cluster caused a shift in parallel application design. Programs using the standard Message Passing Interface (MPI) library ~\cite{Snir:1998:MCR:552013} could not exploit the parallelism on a single node without a rewrite of the underlying algorithms. This resulted in the emergence of hybrid systems that mix MPI and the Open Multi-Processing (OpenMP) ~\cite{openmp08} libraries.  Each node would execute OpenMP program controlled by an MPI process. The OpenMP program then used a fixed number of threads to execute a single work-sharing construct such as a parallel loop ~\cite{gropp2013programming}.

Although the exact exascale ecosystem is unknown, research suggests that data movement will overtake computation as the dominant cost in the system.  This results from the primary growth for parallelism being on chip, with some predictions suggesting hundreds or even thousands of cores per chip die. As a result we will see a higher available bandwidth on chip along with lower latencies for communication within a node.  The lower overhead within a chip provides a significant incentive to develop “communication avoiding” algorithms.  Communication avoidance can occur by re-compute values instead of communicating results when possible as well as by dividing the algorithm into functional partitions where each node solves a different part of an overall application with limited communication necessary between the components.
Many of our current programming models lack the semantics necessary to implement communication avoiding algorithms.  As a result new languages with additional semantics are being proposed for exascale systems.  A common theme among these languages is the ability to statically declare data dependencies and data locality information.  These additional details can then be used by the runtime to aid in scheduling and preemptive data movement.

\subsection{Task-based programming models}

Implemented in several different languages, one type of programming model that may map well onto exascale systems are task-based models.  Task-based models tend to be declarative, an application is broken down into chunks of work and the inputs and outputs to that work are declared in the language semantics.  The explicit data dependencies allow the runtime to optimally schedule and execute the tasks, or chunks of work, in the application.  Execution can often be further improved by the implementation of a secondary file separate from the program that provides hints to the runtime.  The key difference between many task-based models and more traditional programming models is the movement from compute-centric to data-centric application design.  Algorithms are designed around the data a task needs to execute and the data it will produce rather than designed around the computation.

%%% Local Variables: 
%%% mode: latex
%%% TeX-master: "main"
%%% End: 


\section{The CnC Programming Model}

The Concurrent Collections Programming Model (CnC), developed by Intel, is one such data-centric programming model.  Its deterministic semantics allow a task-based runtime to programmatically exploit parallelism.  In addition, it allows for a secondary file, called a tuning spec, to provide additional hints to improve performance.

\begin{figure}[!htb]
  \centering
  \includegraphics[width=0.5\textwidth]{drawings/CnCExample.pdf}
  \caption{CnC Semanticcs Graph}
  \label{fig:cnc_graph}
\end{figure}

The CnC model can be thought of as a producer-consumer paradigm where data is produced and consumed by tasks, or Steps in CnC terminology.  The produced and consumed data is declared explicitly in an input file, known as a CnC graph file.  The steps themselves are also entities that can be produced.  When a step produces another tasks, this is known as control dependencies and is also declared in the CnC graph file.  Figure \ref{fig:cnc_graph} shows a visual representation of a CnC graph file.  The figure shows the steps (rectangles), the data (ovals), and the dependencies between them.  The title of a step is often a verb, describing the work being done within the step.  The title of a data collection is often a noun, describing the data within the collection.  The control dependencies are not shown.  A text based version of the same data, including the control dependencies, is provided to CnC when designing a CnC application, an example of which is given in section 5.

By declaring all dependencies between steps and data the specifics regarding how the algorithm is executed is  abstracted out of the implementation.  As an example, it is clear what data is needed by a given step.  If that data has not been produced yet, the step will not be scheduled.  This allows the runtime to optimally decide when and where to schedule computation. For some more complicated semantic, additional hints can be provided to the runtime through a separate file called a tuning specification.  This is also useful for running the same program on different architectures, as no rewrites of the application are necessary to switch platforms just the tuning specification. 

\subsection{Language Specifics}
The CnC model is built on three key constructs; step collections, data collections, and control collections ~\cite{budimlicconcurrent}. A step collection defines computation, an instance of which consumes and produces data.  The consumed and produced data, or data items, belong to data collections.  Data items within a data collection are indexed using item tags.  Tags can be thought of as tuples and can be anything that can uniquely identify one given instance of the data in the data's collection.  Finally the control collection describes the prescription, or creation, of step instances.  The relationship between these collections as well as the collections themselves are defined in the CnC graph file.

Developing a CnC application then begins with designing the CnC graph file. An algorithm is broken down into computation steps, instances of which correspond to different input arguments. These steps along with the data collections become nodes in a graph. Each step can optionally consume data, produce data, and prescribe additional computation. These relationships, producer, consumer, and control, define the edges in the graph and will dynamically be satisfied as the program executes.

The next and final required step in producing a CnC application is to implement the step logic. The flow within a single CnC step is as follows: consume, compute and produce. This ordering is required as there is no guarantee the data a step needs will be ready when the step begins executing. This is due to steps being preemptively scheduled when they are prescribed.  Most of the time the data will be ready when a steps begins execution, occasionally and often due to implementation error a step’s data may never be available.  Internally if the data is not ready when a CnC step begins execution it will halt execution and try again later. To improve performance, hints can be provided through the tuning specification to increase the likelihood that steps are schedule for execution when their required input data is ready.

\subsection{CnC Example}

\begin{figure}[!htb]
  \centering
  \includegraphics[width=0.5\textwidth]{drawings/FibExample.pdf}
  \caption{CnC Graph: Fibonacci}
  \label{fig:fib_graph}
\end{figure}

% apparently putting this here makes it show up on the top of page 3, where I want it
\begin{figure*}[!htb]
  \centering
  \includegraphics[width=\textwidth]{drawings/CnC.pdf}
  \caption{CnC Graph}
  \label{fig:cnc}
\end{figure*}

As an example, we may consider a simple recursive implementation of the Fibonacci sequence, see figure \ref{fig:fib_graph}.  This application consists of one step, COMPUTE\_FIB, which takes the previous two computed values as input and produces the next value in the sequence.  One data collection, FIB\_DATA exists for the application.  Data within the collection is indexed by a tag consisting of the values sequence number. Tags 1-5, then point to the values 1, 1, 2, 3, 5, respectively.  The first two values of the data collection are produced by the environment, the rest of the values in the collection are produced as needed by COMPUTE\_FIB.  A tag exists for COMPUTE\_FIB as well.  We can index this collection by the integer sequence a particular instance will produce.  For example the step instance at tag 3 will consume the data at tag 1 and 2, and produce data at tag 3.  Specifically it will consume 1, 1 and produce 2.  The number of steps executed in this example is produced by the environment.

%%% Local Variables: 
%%% mode: latex
%%% TeX-master: "main"
%%% End: 


\section{Limitations of Current Ray Tracing Algorithms}

Traditional ray tracing algorithms are embarrassingly parallel as no ray depends on any other ray. The data needed by each individual ray however varies widely as its path is traced. Acceleration structures, such as k-d trees have been developed to increase ray tracing performance. As the data scales up however, it is no longer possible to store an entire data set in an acceleration structure in shared memory. One solution is to implement data decomposition. Each node on a distributed system is then responsible for a subset of the domain. Primary rays and secondary rays are then communicated across nodes as the algorithm executes. These types of models typically rely on expensive preprocessing steps that help to balance both the data distribution and rendering work evenly across nodes ~\cite{navratil2014dynamic}.  

Load balancing, a significant bottle neck of today’s systems may not be when we look towards exascale systems.  The proposed smarter programming models and runtimes will allow for scheduling and data movement decisions to be made at runtime which will help reduce imbalance in a system.  Data and computation can be dynamically migrated off of over worked nodes assuming a properly sized granularity for tasks and data. 


%%% Local Variables: 
%%% mode: latex
%%% TeX-master: "main"
%%% End: 


\section{Exascale Ray Tracing}
\label{sec:raytracing}

When designing our ray tracing prototype for exascale we focused on
two key aspects: (a) producing a task based application with (b) an
emphasis on avoiding communication. For simplicity, we consider here
only ray tracing scenes without reflection or refraction, although our
proposed algorithm can be extended to handle either in the future. Our
algorithm uses a simple voxel decomposition to split the work required
to trace a scene (which we will henceforth refer to as a ``domain'').

Primary camera rays are then sent into the system and propagate
through the domain. As secondary rays introduce most of the uncertainty
in the amount of communication necessary in a ray tracing algorithm,
we introduce a pre-processing technique that distributes light
information to each voxel prior to tracing the domain. This allows the
ray tracing step within each voxel to be completely independent of the
data in the rest of the domain. Using this algorithm we can predict an
upper bound on the amount of communication necessary (a domain that
contains no data), and extrapolate from there rough estimates on how
our algorithm might perform on an exascale system.

\subsection{Implementation}

We chose to implement our ray tracer using Intel’s implementation of
CnC, which is built on top of their Thread Building Blocks (TBB)
library. This runs on today’s multicore systems but has the potential
to do so on anticipated exascale systems.

Figure \ref{fig:cnc} shows the graph for our distributed ray tracer.
It shows data collections, step collections, and the dependencies
between them. The tags corresponding to each collection are also
shown. The control collection for the proposed model is static for all
steps and defined in an initialization step. The graph begins
execution when the object, lights, and camera data are provided, and
terminates when it produces an image.

Let us consider the parts of the graph individually.

\subsection{Tags}

The tags are different for many of the data and step collections but
share common elements. The FRAME tag refers to one specific frame in
the case of an animation. The INSTANCE tag refers to the current
iteration. The I, J, and K tags are iterators over 3D spatial data,
selecting a specific voxel. The RAY\_STEP tag allows for multiple
traversals of the same voxel in the same frame for secondary rays.

\subsection{Data Collections}
\label{sec:datacollections}

The OBJECT data collection contains input data for the domain from the
environment. Currently, this data is extracted from WaveFront
``\texttt{.obj}'' files. The DECOMPOSE\_DOMAIN step collection
partitions this data into voxels, producing the VOXEL\_OBJECT data
collection. Objects that span multiple voxels are duplicated. The
LIGHTS data collection contains data pertaining to light sources. The
VOXEL\_LIGHT data collection contains the same information as LIGHTS
plus a traced light mesh for each wall of a voxel. The CAMERA data
collection contains the location and direction of the camera. The
RAY\_PACKET data collection contains all the rays that intersect a
voxel wall for a given wall and iteration. The IMAGE data collection
contains the final image data.

\subsection{Step Collections}

Recall that these are where the computation is done. They may be
implemented in any programming language CnC supports, which is most of
them.

\subsubsection{DECOMPOSE\_DOMAIN}

As mentions in Section~\ref{sec:datacollections} the DECOMPOSE\_DOMAIN
step takes the data to be traced as input and produces subsets of that
data based on voxel decomposition. As load balancing is not a concern,
a uniformly-gridded voxel decomposition is sufficient. The number of
voxels produced is set at runtime and should be more than the number
of nodes available.
% RRL: Is this really a constraint?

\subsubsection{DISTRIBUTE\_LIGHTS}

In order to reduce the communication of secondary rays,
the DISTRIBUTE\_LIGHTS step is responsible for distributing light
information to each voxel. This guarantees each voxel will only need
to communicate with its direct neighbors. The light information
produced for each voxel contains the original light sources as well as
a light source mesh for each light and each wall of the voxel. The
mesh is produced by tracing rays from each light source to uniformly
spaced points along a voxel wall, see Figure \ref{fig:light}. Where
the rays intersect the wall, new point or directional light sources
are created. If the ray is blocked, the node in the light mesh is
tagged as in shadow.

\begin{figure}[!htb]
\centering
\begin{subfigure}{.49\columnwidth}
 \centering
  \includegraphics[width=.98\columnwidth]{drawings/Lights1.pdf}
  \caption{Initial light rays}
\end{subfigure}
\begin{subfigure}{.49\columnwidth}
 \centering
  \includegraphics[width=.98\columnwidth]{drawings/Lights2.pdf}
  \caption{Point light sources}
\end{subfigure}
\caption{Light Ray Distribution}
\label{fig:light}
\end{figure}

\subsubsection{DISTRIBUTE\_RAYS}

The DISTRIBUTE\_RAYS step is responsible for sending each voxel its
first iteration of ray information. This is an empty set of data for
all voxels except the voxel containing the camera if the camera is
positioned within the domain. If the camera is outside the domain
(e.g. in an orthographic view), multiple voxels may be initialized.

\subsubsection{TRACE\_VOXEL}

The TRACE\_VOXEL step is the heart of the application. This step
executes multiple times for each voxel, depending on the size of the
domain. Each time TRACE\_VOXEL executes, it consumes ray packets from
each of its neighbors. It then traces the rays over its subset of the
domain. If a ray intersects with an object, secondary rays from each
light source are considered if the corresponding point or directional
light source from the voxels light mesh is not in shadow.

Rays that do not intersect objects within the voxels and reach the
voxels walls are collected and passed to the corresponding neighbor on
the next iteration. See Figure~\ref{fig:trace}. Because we are not
considering reflection or refraction, we know the maximum amount of
times we will have to communicate a single ray across voxel borders in
the worst case is proportional to domain size. This allows us to
prescribe the a maximum number of instances of TRACE\_VOXEL in an
initialization step. For the example in Figure~\ref{fig:trace}, that
maximum is 3. As each step will eventually be executed on a single
node of a cluster we plan to implement TRACE\_VOXEL using Embree,
Intel’s ray tracing kernel, in order to optimize per node performance.

\begin{figure}[!htb]
\centering
\begin{subfigure}{.49\columnwidth}
 \centering
  \includegraphics[width=.98\columnwidth]{drawings/Trace1.pdf}
  \caption{Distribute rays}
\end{subfigure}
\begin{subfigure}{.49\columnwidth}
 \centering
  \includegraphics[width=.98\columnwidth]{drawings/Trace2.pdf}
  \caption{Trace voxel; ray step 1}
\end{subfigure}
\begin{subfigure}{.49\columnwidth}
 \centering
  \includegraphics[width=.98\columnwidth]{drawings/Trace3.pdf}
  \caption{Trace voxel; ray step 2}
\end{subfigure}
\begin{subfigure}{.49\columnwidth}
 \centering
  \includegraphics[width=.98\columnwidth]{drawings/Trace4.pdf}
  \caption{Trace voxel; ray step 3}
\end{subfigure}
\caption{Trace Voxel}
\label{fig:trace}
\end{figure}

\subsubsection{PRODUCE\_IMAGE}
When all steps in TRACE\_VOXEL have completed, the final set of ray
packets is produced. Each ray in that packet contains the information
that may be necessary to produce the final image. Merging these is the
responsibility of the PRODUCE\_IMAGE step.

\subsection{Textual CnC Graph File}
To give a more concrete example of how the CnC graph file is
implemented, we include a simplified version of the textual
representation of TRACE\_VOXEL in Figure~\ref{fig:tracevoxel}. In
addition to the step declaration, the code includes the SCENE and
RAY\_PACKET data collections as well as the control dependencies from
the environment for TRACE\_VOXEL.

% Code..
\begin{figure*}[t!]
  \begin{center}
    
\begin{lstlisting}[basicstyle=\ttfamily]
/******************************************************************************
//* Item collection declarations */

// data for each voxel, produced by decompose_domain
[ voxel_object *voxel : frame, i, j, k ];

// ray data passed to neighbors, produced by camera and trace_voxel
[ ray_packet *rays : frame, ray_step, neighbor, i, j, k ];

/******************************************************************************
//* CnC steps */

( trace_voxel : frame, ray_step, i, j, k )
<- [ voxel: frame, i, j, k],
   [ rays : frame, ray_step  , 0, i  , j  , k   ] $when(i<#voxels_i-1),
   [ rays : frame, ray_step  , 1, i  , j  , k   ] $when(i>0),
   [ rays : frame, ray_step  , 2, i  , j  , k   ] $when(j<#voxels_j-1),
   [ rays : frame, ray_step  , 3, i  , j  , k   ] $when(j>0),
   [ rays : frame, ray_step  , 4, i  , j  , k   ] $when(k<#voxels_k-1),
   [ rays : frame, ray_step  , 5, i  , j  , k   ] $when(k>0)
-> [ rays : frame, ray_step+1, 0, i-1, j  , k   ] $when(i>0),
   [ rays : frame, ray_step+1, 1, i+1, j  , k   ] $when(i<#voxels_i-1),
   [ rays : frame, ray_step+1, 2, i  , j-1, k   ] $when(j>0),
   [ rays : frame, ray_step+1, 3, i  , j+1, k   ] $when(j<#voxels_j-1),
   [ rays : frame, ray_step+1, 4, i  , j  , k-1 ] $when(k>0),
   [ rays : frame, ray_step+1, 5, i  , j  , k+1 ] $when(k<#voxels_k-1);

/******************************************************************************
//* Input output relationships from environment */

( $initialize: () )
-> (trace_voxel : $range(0, #num_frames), $range(0, #ray_steps),
           $range(0, #voxels_i), $range(0, #voxels_j), $range(0, #voxels_k));

\end{lstlisting} 
  \end{center}
  
  \caption{The TRACE\_VOXEL Section of the CnC Graph File. This has
    been somewhat simplified for readability.}
  \label{fig:tracevoxel}
\end{figure*}

Under the \emph{Item collection declaration} section we see two item
collections being declared. Once for the domain and one for the ray
packets. Instances of the domain indexed using frame, i, j, and k
which correspond to the frame in the case of an animation and the
3-dimensional identifier for a given voxel. Ray packets are indexed
similarly but also include an entry for the current ray step as well
as a neighbor identifier.

Under \emph{CnC steps} we see the declaration for TRACE\_VOXEL. Each
TRACE\_VOXEL step is indexed using the frame, the voxels i, j, k and
the current ray step. The specific instances of the domain and ray
data consumed and produced by TRACE\_VOXEL can be declared using the
RAY\_STEP tag. The step will always consume its scene data. It may
then consume an incoming ray packet and/or produce an outgoing ray
packet for each of the voxel's interior walls.

Under \emph{Input output relationships from environment} we see the
control for TRACE\_VOXEL. In the initialize step we will produce an
instance of TRACE\_VOXEL for every frame in our animation, for every
step in our ray steps, and for each voxel in our scene's
decomposition.

%%% Local Variables: 
%%% mode: latex
%%% TeX-master: "main"
%%% End: 


\input{sec6_preliminary_results}

\section{Conclusion}
We have taken the first steps in designing and implementing a ray algorithm for exascale.  Although we do not have exascale hardware yet, we are hoping to emulate our implementation on todays systems.  Integrating our CnC graph file with Embree to produce a fully funcitonal distributed ray tracer will be the focus of our future work, but we also hope to explore other aspecs of exascale such as hybrid systems and examine how they may impact our design.  

%%% Local Variables: 
%%% mode: latex
%%% TeX-master: "main"
%%% End: 


%% if specified like this the section will be ommitted in review mode
\acknowledgements{
The authors would like to thank the CnC team at RICE University for their CnC guidance as well as Intel for their implementation of both CnC and Embree.
}
\bibliographystyle{abbrv}

%%use following if all content of bibtex file should be shown
%\nocite{*}
\bibliography{bibliography}{}

\end{document}
