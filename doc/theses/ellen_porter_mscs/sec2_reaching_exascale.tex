%\section{Reaching Exascale}
\label{sec:reaching}

Until 2004, performance of single-core microprocessors increased as
predicted as a result of smaller and faster transistors being
developed (i.e. Moore's Law). At that time, this trend shifted as we
reached an inflection point caused by a chip’s power dissipation
~\cite{kogge2013exascale}. Unable to sufficiently and inexpensively
cool a chip, chip designers looked for other ways to increase
performance. This came in the form of multi-core processors, which are
now the building blocks of many HPC (and other) systems.

The introduction of multi-core processors on each node of a cluster
caused a shift in parallel application design. Programs using the
cross-platform standard Message Passing Interface (MPI) library
~\cite{Snir:1998:MCR:552013} could not efficiently exploit parallelism
on individual nodes without a rewrite of the underlying algorithms.
The Open Multi-Processing (OpenMP) ~\cite{openmp08} library presented
a cross-platform standard for parallel programming on multicore nodes,
which led to the emergence of hybrid systems that mixed MPI and
OpenMP. The cluster would run a collection of MPI processes, one per
node, and each node would then execute an OpenMP program redesigned
from the original single-threaded program which used a fixed number of
threads to execute a single work-sharing construct, such as a parallel
loop ~\cite{gropp2013programming}.

Although the exact form of an exascale ecosystem is unknown, research
suggests that data movement will overtake computation as the dominant
cost in the system.
% RRL: It would be nice to cite something here.
This results from the primary means to increase parallelism is
expected to be on-chip, with some predictions
% RRL: citation?
suggesting hundreds or even thousands of cores
per chip die.
As a result, we will would see a higher available bandwidth on
chip along with lower latencies for communication within a node.
% RRL: It is not logical that lower latency should lead to a need to
%   reduce commmunication, *unless* we're talking about off-chip
%   communication.
The
lower overhead within a chip provides a significant incentive to
develop ``communication avoiding'' algorithms.

Two means to avoid communication are, first, to re-compute values
instead of communicating results when possible and, second, to take
account of the need to minimize communication when partitioning the
algorithm into parallel functional units.

Many of our current programming models lack the semantics necessary to
implement communication-avoiding algorithms. As a result, new
languages with additional semantics are being proposed for exascale
systems. A common theme among these languages is the ability to
statically declare data dependencies and data locality information.
These additional details can then be used by the runtime to aid in
scheduling and anticipatory data movement.

\section{Task-Based Programming Models}

One class of programming model that may map well onto exascale systems
are task-based models. They tend to be declarative: An application is
broken down into chunks of work and the inputs and outputs to that
work are declared in the language semantics. Their explicit data
dependencies allow the runtime to optimally schedule and execute the
tasks, or chunks of work, in the application.

Execution can often be further improved by the implementation of a
secondary specification (a file, typically) separate from the program
that provides ``hints'' to the runtime. The key difference between many
task-based models and more traditional programming models is the
movement from compute-centric to data-centric application design.
Algorithms are designed around the data a task needs to execute and
the data it will produce rather than designed around the computation.

%%% Local Variables: 
%%% mode: latex
%%% TeX-master: "main"
%%% End: 
