%\section{Introduction} % might not need to repeat this as a section
\label{sec:introduction}
  
"Ray tracing is the future and always will be". [TODO: look up quote, use idea 
as first introduction paragraph]
  
Achieving the performance expected from an exascale computer will
require modifications to current hardware architecture which will in
turn affect programming models and runtime\footnote{ %
  We use the term ``runtime'' in the sense of a library or libraries
  compiled into and running as part of an application which is not
  specific to the application but which moderates its interface (e.g.
  memory management, thread prioritization, etc.) with the operating
  system. It's not just a ``library'', as it may have its own threads
  or other execution units. %
} design. Until recent years, performance increased in keeping with
Moore's ``Law'' (which is really more of an observation): The number
of transistors within an integrated circuit doubled approximately
every two years. As we reached a limit on the number of transistors a
single chip could contain, hardware architects had to look for other
ways to keep up with performance advancement expectations. In most
cases, this involved a greater emphasis on parallelism. Consequently,
in order to take advantage of hardware advances, applications,
runtimes, and programming models have often required redesign, if not
reimplementation.

As we look towards the next generation of high-performance computing
(HPC) systems, a shift in application design is again anticipated,
this time to reach exascale performance. On-chip parallelism along
with reduced data movement will be critical for applications to make
optimal use of the hardware and minimize power consumption.

Unfortunately, conventional language semantics will not be sufficient
to exploit the architectural advances being developed such as
inter-core message queues. Therefore, new parallel programming models
and smarter runtimes are being designed. The majority of these models
are ``data-centric'' rather than ``compute-centric'': They allow, for
instance, the runtime scheduler to prioritize scheduling computation
on nodes or cores where the required data already resides rather than
% RRL: Can we standardize on (flaxible) OpenCL nomenclature for
% parallelism?
the next available processor ~\cite{kogge2013exascale}. This kind of
model will reduce communication which is the predicted bottle neck for
exascale systems.

The data produced as output from HPC applications such as fluid
simulations or finite-element models tends to scale in size with
compute power. This is expected to occur with exascale systems as well
and has produced a need for visualization algorithms that can take
advantage of distributed systems as well as an opportunity to design
algorithms that can be integrated into HPC applications to produce
results during execution. Section~\ref{sec:sec5_cnc_ray_tracing_implementation} 
proposes one such design for ray tracing, a commonly used rendering technique,
using the Intel Concurrent Collections (CnC) programming model.

The rest of this paper is organized as follows: We start with a description of 
exascale along with a description of the projected trends in programming models 
that will perform well on exascale.  We then explore one programming model, CnC, 
that is expected to map well to exascale systems.  After describing the CnC 
programming model we analyze current ray tracing algorithms and propose places 
for improvement for exascale.  Specifically, we look at ways we can reduce 
communication overhead within the algorithm.  We then describe the 
implementation details of a ray tracer developed in CnC and look at how it might 
perform on future exascale hardware.  Finally we conclude with a section on 
future work.

\section{Exascale}
\label{sec:exascale}

Until 2004, performance of single-core microprocessors increased as
predicted as a result of smaller and faster transistors being
developed (i.e. Moore's Law). At that time, this trend shifted as we
reached an inflection point caused by a chip’s power dissipation
~\cite{kogge2013exascale}. Unable to sufficiently and inexpensively
cool a chip, chip designers looked for other ways to increase
performance. This came in the form of multi-core processors, which are
now the building blocks of many HPC (and other) systems.

The introduction of multi-core processors on each node of a cluster
caused a shift in parallel application design. Programs using the
cross-platform standard Message Passing Interface (MPI) library
~\cite{Snir:1998:MCR:552013} could not efficiently exploit parallelism
on individual nodes without a rewrite of the underlying algorithms.
The Open Multi-Processing (OpenMP) ~\cite{openmp08} library presented
a cross-platform standard for parallel programming on multicore nodes,
which led to the emergence of hybrid systems that mixed MPI and
OpenMP. The cluster would run a collection of MPI processes, one per
node, and each node would then execute an OpenMP program redesigned
from the original single-threaded program which used a fixed number of
threads to execute a single work-sharing construct, such as a parallel
loop ~\cite{gropp2013programming}.

Although the exact form of an exascale ecosystem is unknown, research
suggests that data movement will overtake computation as the dominant
cost in the system.
% RRL: It would be nice to cite something here.
This results from the primary means to increase parallelism is
expected to be on-chip, with some predictions
% RRL: citation?
suggesting hundreds or even thousands of cores
per chip die.
As a result, we will would see a higher available bandwidth on
chip along with lower latencies for communication within a node.
% RRL: It is not logical that lower latency should lead to a need to
%   reduce communication, *unless* we're talking about off-chip
%   communication.
The
lower overhead within a chip provides a significant incentive to
develop ``communication avoiding'' algorithms.

Two means to avoid communication are, first, to re-compute values
instead of communicating results when possible and, second, to take
account of the need to minimize communication when partitioning the
algorithm into parallel functional units.

Many of our current programming models lack the semantics necessary to
implement communication-avoiding algorithms. As a result, new
languages with additional semantics are being proposed for exascale
systems. A common theme among these languages is the ability to
statically declare data dependencies and data locality information.
These additional details can then be used by the runtime to aid in
scheduling and anticipatory data movement.

\section{Ray tracing}
\label{sec:raytracing}

Ray tracing is one of the rendering techniques often used in computer graphics 
to render three-dimensional scenes into two dimensional images [Shirley].  
A ray tracing renderer takes a set of objects in 3D space as input and then 
casts viewing rays into the scene to determine the color of each pixel 
in an output image.  

As outlined in Shirley [ref], ray tracing has three main components, ray 
generation, ray intersection and shading.  The first component is responsible
for computing viewing rays, which are rays from an origin position to a point on 
an image plane.  An image plane is the plane that will contain the final output 
image; it is positioned between the eye, or origin, location and the scene to be 
rendered. [graphic would be good here?].

Each viewing ray is then cast into the scene where we apply the second component, 
ray intersection.  For each ray we need to know what object in the scene it 
intersects first.  This tells us which object can be seen by that viewing ray,
allowing us to color the pixel of the image that the viewing ray passed through
by shading, the third component, our intersected object.  Most shading models 
require information from secondary rays in order to compute the correct color.  
These secondary rays include light rays which directional rays pointing from 
light sources towards the intersection point as well as reflected and refracted 
rays depending on the type of material of the object at the intersection point.

%%% Local Variables: 
%%% mode: latex
%%% TeX-master: "main"
%%% End: 
