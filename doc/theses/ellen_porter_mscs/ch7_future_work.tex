\section{Future Work}
\label{sec:future-work}

This thesis presents a base system for task-based ray tracing that emphasizes 
communication reduction.  Several modifications to the algorithm could be made 
to improve runtime performance and feature support.  These improvements include
modifications to support GPU and CPU ray tracing, optimizing and future
evaluation on the implemented light meshes used, implementation of reflector
material meshes or another algorithm for reducing communication due to reflected
rays, and implementation of integrated restart functionality to support ray
refraction.  

\subsection{GPU ray tracing}
The exact landscape for future exascale distributed systems is unknown, but it 
has been suggested that the systems may have a combination of GPUs as well of 
CPUs to allow algorithms to take advantage of different hardware depending on 
their needs~\cite{kogge2013exascale}.  Our current ray tracing algorithm is 
integrated  with Embree, an optimized ray tracing library built for CPU�s.  
Embree could be swapped out for NVidia�s OptiX~\cite{parker2010optix}, a GPU 
based ray tracing engine.  Some scenes may show increased performance when ray 
traced on a GPU over a CPU and would benefit for the library swap.  In addition,
in the case of multiple viewpoints being rendered for a single static scene, 
there is the potential to detect and migrate data at runtime to the specific 
hardware a voxels data-set matches to improve runtime performance.  Exploring 
this realm of runtime optimizations is left for future work.

\subsection{Reflector material meshes}
Implementation and evaluation of the use of reflector meshes to reduce
communication is another area left for future work.  The concept introduced in 
section~\ref{sec:ca-ray-tracing} is similar to light meshes, introduced in the 
same section and detailed in section~\ref{sec:distribute-lights}.
Reflector material meshes could be used to eliminate the communication cost
associated with reflection, but come at a potentially large memory overhead
cost.  We will look at the details of creating and using a reflector mesh, then
evaluate the potential down-sides.

Pseudo code for creating a reflector mesh can be found in 
figure~\ref{fig:ray_tracing_2} b.  For a given reflective surface the first step 
would be to create a sample of points across the surface.  For each sampled 
point, a distribution of rays would be generated.  Each ray would originate from
the sample point and point in a different direction. The density of the surface
points and the distributed rays would need to be evaluated to determine optimal
sample sizes.  

Each of the distributed rays would then be cast from their origin into the
scene.  This casing could use a algorithm similar to the ray casting algorithm
presented in section ~\ref{sec:ca-ray-casting}. All distributed rays would be 
given to every voxel, if the ray intersects an object, material information from 
the intersected object would be stored, otherwise no information would be set.  
Once all voxels had computed the intersections, the results would be composed 
into a single data set we are calling a reflected material mesh.  This mesh 
would contain all the distribution rays for each sample point and each 
distribution ray would contain material information from the first object 
intersected.

The reflected material meshes would be passed to each voxel computation at
runtime to be used by the ray tracing algorithm.  When a ray intersected a
reflector, a quick lookup would give the appropriate mesh for the algorithm to
use.  A nearest neighbor approach could then be used to find the closest sample
point or points.  The direction of the reflected ray would need to be used to
determine which ray from the sample points distributed ray set should be used.
Once the ray is selected, the material information it contains could then be
used to compute the intersection color on the reflected material.

The direction of the reflected ray greatly affects the object that will be
reflected on a surface.  This results in the reflector material meshes differing
significantly from the light meshes.  Light meshes are attached to voxel walls
and samples with a single direction are sufficient to approximate illumination
calculations, see section~\ref{sec:trace-voxel}.  Reflector material meshes must
be sample and attached to every reflector surface.  In addition each point
requires potentially many rays to be evaluated.  This may significantly increase
the pre-processing cost as well as the memory overhead.  In a worst case
scenario a scene may have many reflectors all hidden behind a wall.  The
computation to compute the meshes would be done, but none of the viewing rays
would ever use the information, resulting in wasted memory and computation
space.  Optimizations and viability of this design as well as additional
communication reducing algorithms designed for handling reflected rays are left
to future work. 

\subsection{Refracted rays}
Extending our ray tracer to support refracted rays while keeping communication
costs low presents an opportunity to exploit another anticipated architecture
feature of exascale distributed systems.  Hardware failures are anticipated to
be more common at exascale~\cite{gropp2013programming}.  As a result, 
runtimes will need to detect failures and be able to recover from them.  One way
to do this is to periodically send memory information to a file and then use the
file to recover.  This capability could be leveraged by the application to 
manually invoke a failure when the system processes refraction.  The voxel 
computing incorrect data using non-refracted rays would be stopped.  The new 
trajectories of the refracted rays would be computed and the downstream voxel 
computations would be restarted with the new ray directions.































