\section{Future Work}
\label{sec:future-work}

This thesis presents a base system for task-based ray tracing that emphasizes 
communication reduction.  Several modifications to the algorithm could be made 
to improve runtime performance and feature support.  These improvements include
modifications to support GPU and CPU ray tracing, support for reflected and
refracted rays and support for soft shadows.

\subsection{Reflected and Refracted Rays}
As introduced in Chapter~\ref{chpt:design}, extending our ray tracer to support
reflected and refracted rays while keeping communication costs low presents an
opportunity to exploit another anticipated architecture feature of exascale
distributed systems.  Hardware failures are anticipated to be more common at
exascale~\cite{article-gropp}.  As a result, runtimes will need to
detect failures and be able to recover from them.  Two ways to do this are to
periodically send memory information to a file or duplicate the information
across nodes.  This file or secondary node can then be used to recover when a
failure is detected. This capability could be leveraged by the application to
manually invoke a failure when the system detects refraction while tracing a
voxel.  The node computing incorrect results using non-reflected or
non-refracted rays would be stopped.  The new trajectories of the rays would be
computed and the downstream voxel computations would be restarted with the new
ray directions.

\subsection{Soft Shadows}
The implemented ray tracing algorithm creates hard lines along the edge of
shadow.  Techniques have been developed to produce soft shadows for area light
sources by casting multiple light rays from each intersection point that
point to different locations along the area light source. The results of these
are combine to produce a final value representing how ``in shadow'' a particular
pixel should be.  This creates the illusion of a softer line along the shadows
edge. The use of the shadow mesh presents an opportunity to evaluate how ``in
shadow'' a pixel is by using the surrounding vertices in the mesh if some are in
shadow and some are not.

\subsection{GPU Ray Tracing}
The exact landscape for future exascale distributed systems is unknown, but it 
has been suggested that the systems may have a combination of GPUs as well of 
CPUs to allow algorithms to take advantage of different hardware depending on 
their needs~\cite{article-kogge}.  Our ray tracing algorithm is integrated 
with Embree, an optimized ray tracing library built for CPU�s. Embree could be
swapped out for NVIDIA's OptiX~\cite{proceedings-parker}, a GPU-based ray tracing
engine.  Some scenes may show increased performance when ray traced on a GPU
over a CPU and would benefit for the library swap.  In addition, in the case of
multiple viewpoints being rendered for a single static scene, there is the
potential to detect and migrate information at runtime to the specific hardware
a voxel's subset of the scene executes best on to improve runtime performance. 
Exploring this realm of runtime optimizations is left for future work.





























