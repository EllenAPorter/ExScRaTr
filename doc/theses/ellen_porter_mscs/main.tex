\documentclass[MSc,12pt]{wsuthesis}

%\usepackage[hdvipdfm]{graphics}
\usepackage[table]{xcolor}
\usepackage{verbatim}
\usepackage{graphicx}
\usepackage{color}
\usepackage{pstricks}
\usepackage{setspace}
\usepackage{url}
\usepackage{listings}
\usepackage{appendix}
\usepackage{lscape}
\usepackage{graphicx}
\usepackage{times}
\usepackage{cite} 
\usepackage{subcaption}
\usepackage{listings} 
\usepackage{url}
\usepackage{changepage}
\usepackage{multirow}

% import figures
%\input{figures/figures.tex}

\pdfpxdimen=1in
\divide\pdfpxdimen by 600
\setlength{\fboxsep}{0pt}
\setlength{\fboxrule}{2px}
\definecolor{wsu-gray}{RGB}{113,113,113}
\definecolor{wsu-light-gray}{RGB}{215,218,219}


\lstnewenvironment{algorithm}[1][] %defines the algorithm listing environment
{   
    \lstset{ %this is the stype
        frame=tB,
        numbers=left, 
        showlines=true,
        numberstyle=\tiny,
        basicstyle=\scriptsize, 
        keywordstyle=\color{black}\bfseries,
        keywords={,input, output, return, datatype, function, in, if, else, 
                   foreach, while, begin, end, for, all, do, out, all, then, is,
                   not, each, }
        numbers=left,
        xleftmargin=.04\textwidth,
        #1 
    }
}{}

\lstnewenvironment{cnc}[1][] %defines the algorithm listing environment
{   
    \lstset{ %this is the stype
        frame=tB,
        numbers=left, 
        showlines=true,
        numberstyle=\tiny,
        basicstyle=\scriptsize, 
        keywordstyle=\color{black}\bfseries,
        keywords={, \$context, \$range, \$initialize, \$finalize, (, ),
        [, ], )), } numbers=left,
        xleftmargin=.04\textwidth,
        #1 
    }
}{}

%\newcommand\fnurl[2]{%
%  \href{#2}{#1}\footnote{\url{#2}}%
%}

\begin{document}

\title{Communication Avoiding Ray Tracing \\ for Exascale Computing}

\author{Ellen Alice Porter}

\submitdate{July 2016}

\centerhdrstrue
\strhdrstrue

\dept{School of Electrical Engineering and Computer Science}

\chair{Robert R. Lewis}

\acknowledgment{
  The completion of this thesis would not have been possible without the help of
  many people.  I would like to express my sincerest gratitude to my advisor,
  Dr. Robert R. Lewis for his guidance and patience over the years.  I would
  also like to acknowledge my thesis committee including Dr. Wayne O.
  Cochran and Dr. Carl H. Hauser for their commitment towards my education. The
  CnC community was essential to the completion of this research, especially
  Kath Knobe and Nick Vrvilo, thank you both.  I must also recognize my
  employer, Battelle, my managers and my colleagues for their support and
  encouragement.  Last but not least, I would like to thank my family,
  especially Mathieu, for their unconditional faith, inspiration and love.
}

\abstract{
  Exascale computers, defined as being capable of performing at least
  one exaFLOP ($10^{18}$ floating point operations per second) are
  anticipated to emerge in the next several years. Reaching this scale
  of computation will require hardware and software changes for
  high-performance computing (HPC). Applications will need to adapt as
  the architecture of supercomputers change. Some studies suggest
  \emph{communication avoiding} algorithms might be the most performant
  design for future systems. This has created an interest in the
  development of scalable visualization algorithms and techniques.

  This thesis explores the impact exascale hardware will have on
  programming models and application design. We look at one
  specific programming model, Concurrent Collections (CnC), and explore how we
  can use it coupled with Intel's ray tracing engine, Embree, to build a
  scalable ray tracing system with an emphasis on communication avoidance and
  extension for exascale. %
}

\dedication{
  \emph{For my mom and dad.}
}

\preface

\chapter{Introduction}
%\section{Introduction} % might not need to repeat this as a section
\label{sec:introduction}
  
"Ray tracing is the future and always will be". [TODO: look up quote, use idea 
as first introduction paragraph]
  
Achieving the performance expected from an exascale computer will
require modifications to current hardware architecture which will in
turn affect programming models and runtime\footnote{ %
  We use the term ``runtime'' in the sense of a library or libraries
  compiled into and running as part of an application which is not
  specific to the application but which moderates its interface (e.g.
  memory management, thread prioritization, etc.) with the operating
  system. It's not just a ``library'', as it may have its own threads
  or other execution units. %
} design. Until recent years, performance increased in keeping with
Moore's ``Law'' (which is really more of an observation): The number
of transistors within an integrated circuit doubled approximately
every two years. As we reached a limit on the number of transistors a
single chip could contain, hardware architects had to look for other
ways to keep up with performance advancement expectations. In most
cases, this involved a greater emphasis on parallelism. Consequently,
in order to take advantage of hardware advances, applications,
runtimes, and programming models have often required redesign, if not
reimplementation.

As we look towards the next generation of high-performance computing
(HPC) systems, a shift in application design is again anticipated,
this time to reach exascale performance. On-chip parallelism along
with reduced data movement will be critical for applications to make
optimal use of the hardware and minimize power consumption.

Unfortunately, conventional language semantics will not be sufficient
to exploit the architectural advances being developed such as
inter-core message queues. Therefore, new parallel programming models
and smarter runtimes are being designed. The majority of these models
are ``data-centric'' rather than ``compute-centric'': They allow, for
instance, the runtime scheduler to prioritize scheduling computation
on nodes or cores where the required data already resides rather than
% RRL: Can we standardize on (flaxible) OpenCL nomenclature for
% parallelism?
the next available processor ~\cite{kogge2013exascale}. This kind of
model will reduce communication which is the predicted bottle neck for
exascale systems.

The data produced as output from HPC applications such as fluid
simulations or finite-element models tends to scale in size with
compute power. This is expected to occur with exascale systems as well
and has produced a need for visualization algorithms that can take
advantage of distributed systems as well as an opportunity to design
algorithms that can be integrated into HPC applications to produce
results during execution. Section~\ref{sec:sec5_cnc_ray_tracing_implementation} 
proposes one such design for ray tracing, a commonly used rendering technique,
using the Intel Concurrent Collections (CnC) programming model.

The rest of this paper is organized as follows: We start with a description of 
exascale along with a description of the projected trends in programming models 
that will perform well on exascale.  We then explore one programming model, CnC, 
that is expected to map well to exascale systems.  After describing the CnC 
programming model we analyze current ray tracing algorithms and propose places 
for improvement for exascale.  Specifically, we look at ways we can reduce 
communication overhead within the algorithm.  We then describe the 
implementation details of a ray tracer developed in CnC and look at how it might 
perform on future exascale hardware.  Finally we conclude with a section on 
future work.

\section{Exascale}
\label{sec:exascale}

Until 2004, performance of single-core microprocessors increased as
predicted as a result of smaller and faster transistors being
developed (i.e. Moore's Law). At that time, this trend shifted as we
reached an inflection point caused by a chip’s power dissipation
~\cite{kogge2013exascale}. Unable to sufficiently and inexpensively
cool a chip, chip designers looked for other ways to increase
performance. This came in the form of multi-core processors, which are
now the building blocks of many HPC (and other) systems.

The introduction of multi-core processors on each node of a cluster
caused a shift in parallel application design. Programs using the
cross-platform standard Message Passing Interface (MPI) library
~\cite{Snir:1998:MCR:552013} could not efficiently exploit parallelism
on individual nodes without a rewrite of the underlying algorithms.
The Open Multi-Processing (OpenMP) ~\cite{openmp08} library presented
a cross-platform standard for parallel programming on multicore nodes,
which led to the emergence of hybrid systems that mixed MPI and
OpenMP. The cluster would run a collection of MPI processes, one per
node, and each node would then execute an OpenMP program redesigned
from the original single-threaded program which used a fixed number of
threads to execute a single work-sharing construct, such as a parallel
loop ~\cite{gropp2013programming}.

Although the exact form of an exascale ecosystem is unknown, research
suggests that data movement will overtake computation as the dominant
cost in the system.
% RRL: It would be nice to cite something here.
This results from the primary means to increase parallelism is
expected to be on-chip, with some predictions
% RRL: citation?
suggesting hundreds or even thousands of cores
per chip die.
As a result, we will would see a higher available bandwidth on
chip along with lower latencies for communication within a node.
% RRL: It is not logical that lower latency should lead to a need to
%   reduce communication, *unless* we're talking about off-chip
%   communication.
The
lower overhead within a chip provides a significant incentive to
develop ``communication avoiding'' algorithms.

Two means to avoid communication are, first, to re-compute values
instead of communicating results when possible and, second, to take
account of the need to minimize communication when partitioning the
algorithm into parallel functional units.

Many of our current programming models lack the semantics necessary to
implement communication-avoiding algorithms. As a result, new
languages with additional semantics are being proposed for exascale
systems. A common theme among these languages is the ability to
statically declare data dependencies and data locality information.
These additional details can then be used by the runtime to aid in
scheduling and anticipatory data movement.

\section{Ray tracing}
\label{sec:raytracing}

Ray tracing is one of the rendering techniques often used in computer graphics 
to render three-dimensional scenes into two dimensional images [Shirley].  
A ray tracing renderer takes a set of objects in 3D space as input and then 
casts viewing rays into the scene to determine the color of each pixel 
in an output image.  

As outlined in Shirley [ref], ray tracing has three main components, ray 
generation, ray intersection and shading.  The first component is responsible
for computing viewing rays, which are rays from an origin position to a point on 
an image plane.  An image plane is the plane that will contain the final output 
image; it is positioned between the eye, or origin, location and the scene to be 
rendered. [graphic would be good here?].

Each viewing ray is then cast into the scene where we apply the second component, 
ray intersection.  For each ray we need to know what object in the scene it 
intersects first.  This tells us which object can be seen by that viewing ray,
allowing us to color the pixel of the image that the viewing ray passed through
by shading, the third component, our intersected object.  Most shading models 
require information from secondary rays in order to compute the correct color.  
These secondary rays include light rays which directional rays pointing from 
light sources towards the intersection point as well as reflected and refracted 
rays depending on the type of material of the object at the intersection point.

%%% Local Variables: 
%%% mode: latex
%%% TeX-master: "main"
%%% End: 
 % subsections: introduction, exascale, ray tracing
  
\chapter{Previous Work} % or literature review/background?
When designing algorithms for distributed systems it is important to consider 
concurrency in algorithm design.  After providing a brief definition of parallel
and distributed computing, we therefore look at Petri nets, a mathematical 
modeling language that has been used to describe algorithms designed for 
distributed systems.  Expanding on this idea, we next look at task-based 
programming models which often include designing for concurrent execution in
their language definition.  They are also anticipated to map well to future 
exascale system design.  CnC, developed by Intel, is one such task-based model 
that we will look at.  This chapter concludes with an outline of the current 
state of distributed and parallel ray tracing along with details on Embree, a
parallel ray tracing library developed by Intel.

\section{Parallel and Distributed Computing}
\label{computing}

Parallel computing is a straight forward concept where a single task is broken 
down into into smaller sub tasks.  Each of the sub tasks can then be executed at
the same time, or in parallel.  Once a task completes it combines its results 
with the result of other complete tasks until all sub tasks are complete and the
solution to the initial task is found.  In an ideal situation, the time to 
complete the initial task scales proportionally to the number of sub tasks 
created.  In practice, overhead which includes the creation of sub tasks, 
the communication between the tasks as they execute and the final aggregation of 
the results hinders this speed up.

Parallel computing is accomplished on parallel architectures.  This usually 
refers to a single machine, with one or more processors that can all access and
share the machines memory.  A distributed system refers to a collection of
machines, possibly parallel machines, all working together.  The individual 
machines in the system are often called nodes and have access only to their 
machines memory.  To share data, the nodes must communicate with each other
through a communication channel. This is slower than directly accessing shared
memory.  The communication overhead must be considered when optimizing an 
algorithm for distributed execution.

\section{Petri Nets}
\label{sec:petri-nets}
One of key challenges algorithm designers face when designing a parallel 
application is that of concurrency.  Due to the often unpredictable order of
execution within an application it is often difficult to detect all errors 
through traditional testing~\cite{franco2012true}.  Petri nets provide a means 
of proving the correctness of a program given concurrent execution.  They also 
map closely to the design of task-based programming models and can therefore be 
used as a basis for designing algorithms for tasked-based applications.

\begin{figure}[!htb]
\minipage{0.32\textwidth}
  \includegraphics[width=\linewidth]{drawings/Petri1.pdf}
  \caption{Waiting for tokens}\label{fig:petri1}
\endminipage\hfill
\minipage{0.32\textwidth}
  \includegraphics[width=\linewidth]{drawings/Petri2.pdf}
  \caption{Transition ready to fire}\label{fig:petri2}
\endminipage\hfill
\minipage{0.32\textwidth}%
  \includegraphics[width=\linewidth]{drawings/Petri3.pdf}
  \caption{After transition fires}\label{fig:petri3}
\endminipage
\caption{Petri net}
\label{fig:petri}
\end{figure}

Petri nets are bipartite directed graphs with two kind of nodes, \emph{places}
and \emph{transitions}.  Connections between the nodes are called \emph{arcs}
and can only connect places to transitions or transitions to places.  Tokens are
held by places and used to represent firing criteria for a transition.  
Figure~\ref{fig:petri} shows a simple example of a Petri net graph.  The 
example defines three places, A, B, and C and one transition.  The firing 
criteria for the transition is two tokens from A and one token from B.  Once 
sufficient tokens arrive, the transition is able to fire, see 
figure~\ref{fig:petri2}.  The transition consumes the two tokens from A and the
one token from B and produces a single token for C, see Figure~\ref{fig:petri3}.

Once formalized into a Petri net, an algorithm can be \emph{unfolded} into an 
occurrence net which represents every specific instance of execution in a flat
linear way.  Once an occurrence net has been defined it can be used to analyze a
concurrent application and prove correctness~\cite{franco2012true}.  We will use 
Petri nets to outline our idealized ray tracing engine for concurrent execution 
on future exascale systems in section~\ref{sec:design}.  

\section{Task-Based Programming Models}
\label{sec:task-based}

One class of programming model that are anticipated to map well onto exascale 
systems are task-based models. They tend to be declarative: An application is
broken down into chunks of work and the inputs and outputs to that work are 
declared in the language semantics. Their explicit data dependencies allow the 
runtime to optimally schedule and execute the tasks, or chunks of work, in the 
application.

Execution can often be further improved by the implementation of a
secondary specification (a file, typically) separate from the program
that provides \emph{hints} to the runtime. The key difference between many
task-based models and more traditional programming models is the
movement from compute-centric to data-centric application design.
Algorithms are designed around the data a task needs to execute and
the data it will produce rather than designed around the computation.

\subsection{The CnC Programming Model}
\label{sec:cnc}

The Concurrent Collections Programming Model (CnC), developed by
Intel, is one such data-centric programming model. Its deterministic
semantics allow a task-based runtime to programmatically exploit
parallelism. In addition, it allows for a secondary file, called a
tuning specification, to provide additional hints that can improve performance.

The CnC model can be thought of as a producer-consumer paradigm where
data is produced and consumed by tasks, or \emph{steps} in CnC
terminology. The produced and consumed data is declared explicitly in
an input file, known as a \emph{graph file}. The steps themselves are
also entities that can be produced. When a step produces another step,
this is known as a \emph{control dependency} and is also declared in the
graph file.

Figure~\ref{fig:cnc_graph} shows a example graph file. The rectangles represent
\emph{step collections}, the ovals represent \emph{data collections} are the
dependencies are the directed edges connecting them. The title of a step
collection is usually a descriptive verb and the title of a data
collection is usually a descriptive noun. The control dependencies are
not shown. A text description of Figure~\ref{fig:cnc_graph} (which
includes control information) is provided to CnC when designing a CnC
application.  Section~\ref{sec:implementation} shows an example of this.

\begin{figure}[t]
  \centering
  \includegraphics[width=0.5\textwidth]{drawings/CnCExample.pdf}
  \caption{CnC Graph Semantics}
  \label{fig:cnc_graph}
\end{figure}

By declaring all dependencies between steps and data, the specifics
regarding how the algorithm is executed is abstracted out of the
implementation. For example: It is clear what data is needed by a
given step, so if that data has not been produced yet, the step will
not be scheduled. This allows the runtime to optimally decide when and
where to schedule computation. By way of contrast, in a conventional
multithreaded program, it is the responsibility of the programmer to
guarantee that a thread's inputs are available and consequently start
the thread.

For some more complicated semantics, additional hints can be provided
to the runtime through a \emph{tuning specification}. As the tuning
specification usually is in a separate file, this makes it easy to run
the same program on different architectures, as no rewrites of the
application are necessary to switch platforms: just the tuning
specification.

\subsubsection{Language Specifics}
\label{sec:cnc_language}

The CnC model is built on three key constructs; step collections, data
collections, and control collections ~\cite{budimlicconcurrent}. A
step collection defines computation, an instance of which consumes and
produces data. The consumed and produced data items belong to data
collections. Data items within a data collection are indexed using
item epmh{tags}: tuples that, like primary keys, can uniquely identify a
data item in the data collection. Finally, the control collection
describes the prescription, or creation, of step instances. The
relationship between these collections as well as the collections
themselves are defined in the graph file.

Developing a CnC application then begins with designing the graph
file. An algorithm is broken down into computation steps, instances of
which correspond to different input arguments. These steps, along with
the data collections become nodes, in the graph. Each step can
optionally consume data, produce data, and/or prescribe additional
computation. These relationships: producer, consumer, and control,
define the edges of the graph and will dynamically be satisfied as the
program executes.

The next and final required step in producing a CnC application is to
implement the step logic. The flow within a single step is: consume,
compute, and produce. This ordering is required as there is no
guarantee the data a step needs will be ready when the step begins
executing. This is due to steps being preemptively scheduled when they
are prescribed. Most of the time the data \emph{will} be ready when a
step begins execution, but occasionally and often due to an
implementation error, a step's data may never be available.
Internally, if the data is not ready when a step begins execution
it will halt execution and try again later. To improve performance,
hints can be provided through the tuning specification to increase the
likelihood that steps are schedule for execution when their required
input data is ready.

\subsection{Example}
\label{sec:cnc_example}

\begin{figure}[!tb]
  \centering
  \includegraphics[width=0.5\textwidth]{drawings/FibExample.pdf}
  \caption{A CnC Graph to Compute the Fibonacci Sequence}
  \label{fig:fib_graph}
\end{figure}

Figure~\ref{fig:fib_graph} shows an example of a
simple iterative implementation of the Fibonacci sequence. This
application consists of one step, COMPUTE\_FIB, which takes the
previous two computed values as input and produces the next value in
the sequence. One data collection, FIB\_DATA, exists for the
application. Data within the collection is indexed by a tag consisting
of the sequence number. Tags 1-5, then index the values 1,
1, 2, 3, 5, respectively. The first two values of the data collection
are produced by the environment, the rest of the values in the
collection are produced as needed by COMPUTE\_FIB. A tag exists for
COMPUTE\_FIB as well. We can index this collection by the integer
sequence a particular instance will produce. For example, the step
% RRL: What do you mean by "integer sequence"? The *whole* Fib sequence?
instance at tag 3 will consume the data at tag 1 and 2, and produce
data at tag 3. Specifically it will consume 1, 1 and produce 2.
The number of steps executed in this example is provided by the 
environment.

\section{Distributed and Parallel Ray Tracing}
\label{sec:ray_tracing}

\emph{"Ray tracing is the future and ever will be"} was the title of a SIGGRAPH 
course in 2013.  The course outlined the state of the art in ray tracing 
technology, covering recent developments and optimization techniques to speed up 
the core algorithm such as acceleration data structures.  This quote points out 
that although we have optimized ray tracing to a T, it is still not the primary
rendering technique used in graphics.  This section outlines key strategies for 
ray tracing optimization with an emphasis on parallel and distributed 
advancements.

Ray tracing is an application that tends to scale well as you subdivide the 
task.  Each ray cast into a scene does not need any information about any other 
ray cast into the scene.  Communication between rays, therefore, is non-existent 
and only the cost of creating new tasks and joining the tasks to produce the 
final ray traced image limit the amount of parallelism you can extract from the
algorithm\footnote{ %
  The system you are executing a ray tracer on limits the parallelism of the 
  application.  
}.

Although no ray depends on any other ray in a ray tracing algorithm, the data 
needed by each individual ray varies widely as its path is traced. Acceleration 
structures such as k-d trees have been developed to increase ray tracing 
performance. As the size of the scene increases, however, it is no longer 
possible to store an entire data set in an acceleration structure in shared 
memory.

One solution is to implement data decomposition. Each node on a distributed 
system is then responsible for a subset of the domain. Primary, secondary, 
and subsequent rays are then communicated across nodes as the algorithm 
executes. These types of models typically rely on expensive pre-processing steps 
that help to balance both the data distribution and rendering work evenly across 
nodes ~\cite{navratil2014dynamic}.

Load balancing, a significant bottleneck on today's systems, may not be easily 
implemented on exascale systems. The proposed smarter programming models and 
runtimes, on the other hand, will allow for scheduling and data movement 
decisions to be made at runtime which will help reduce imbalance in a system. 
Data and computation can be dynamically migrated off of overworked nodes 
assuming a properly sized granularity for tasks and data.

\section{Embree}
\label{sec:embree}
Embree is a parallel ray tracing library developed at Intel that has been 
optimized at the architecture level to take advantage of chip specific caching 
and communication.  Embree supports both packet (multiple rays) and single ray 
intersection queries into optimized data structures ~\cite{wald2014embree}. The 
data structures used in Embree are BVH or bounding volume hierarchy structures.  
BHV data structures are hierarchical structures that are fast to build with 
small memory footprints and fast traversal times due to their shallow depth 
~\cite{wald2014embree}.  Due to the optimization within Embree, the library 
exploits on node parallelism and is on par with other state of the art ray 
tracing libraries. 

%%% Local Variables: 
%%% mode: latex
%%% TeX-master: "main"
%%% End: 
 % subsections: distributed ray tracing, task-based
                          % programming models, cnc model, petri net

\chapter{Design}
\label{sec:design}

Libraries such as Embree, see section~\ref{sec:embree} provide optimized ray 
tracing algorithms that exploit parallelism on a single machine.  When we look
at larger data sets that will no longer fit into the memory of a single machine,
we may consider the use of distributed systems, section~\ref{sec:computing} to 
render images.  The next generation of distributed systems capable of exascale
computation provides an opportunity and in some cases a necessity to redesign 
current distributed applications.  Two of the main differences that will set
exascale systems apart from current distributed systems is the increase in the
number of nodes available and the reduction of the memory available on a single
node.  For algorithm designers, this means there will be increased communication 
overhead as data will need to be passed more frequently between individual nodes.

Communication avoiding algorithms have been proposed as a way to reduce 
communication overhead.  The basic idea is to re-do computation when possible
instead of communicating results.  We use this idea as a basis for designing a 
distributed ray tracing application.  On node parallelism can be exploited by
taking advantage of Embree or another optimized library.  This leaves the key 
challenge as how to reduce the communication necessary between the nodes of the 
distributed system.

\section{Data Decomposition}
\label{sec:data_decomposition}
Designing a ray tracer for exascale systems becomes an interesting problem once
we consider scenes to render that have too much information to fit entirely into
the memory of a single node.  In these scenes, some type of data decomposition 
is required where the scene is divided into smaller pieces and distributed 
amongst the available nodes. To fully utilize the on node parallelism we would 
want to give each node enough data to fill its available memory. As each scene 
to be rendered is unique the idealized distribution for one scene will not be 
the same as the idealized distribution for another.  This introduces the concept 
of load balancing.

Load balancing is often addressed with a pre-processing step where the domain is 
optimally divided into evenly sized chunks of data before ray tracing beings. 
As the pre-processing step is often computationally expensive, the expense must 
be considered and weighed against alternative designs. For example, a naive 
alternative approach is to divide the space based on a uniform spatial 
distribution.  This results in unevenly balanced chunks of data but takes little
time to compute.  In this naive approach where the scene has not been balanced 
it is likely to end up with nodes that have lots of work while other nodes have 
little work.

Determining the optimal design pattern for data decomposition becomes a question
of whether the cost of pre-processing outweighs the cost of an unbalanced 
system.  Assuming the pre-processing step were free, computationally speaking, 
then the solution would be to use it.  This would ensure the minimum number of
nodes are used and that each one is used to its full potential, memory wise.  
However since there is a cost to pre-processing we must consider algorithms that
reduce this cost.  

Dividing the domain uniformly in space for example reduces the pre-processing 
cost but requires the use of more nodes then would be needed by a load balanced
distribution.  In addition, each node may or may not fill the memory available. 
Since each node need not be responsible for a single chunk of the domain, it is 
possible of offset this concern and increase memory usage per node by assigning 
each node multiple chunks of data to each node.

\begin{figure}[!htb]
\centering
  \includegraphics[height=5cm]{drawings/sanmiguel_cam25.pdf}
  %\footnote{
  %\href{http://www.pbrt.org/scenes_images/sanmiguel\_cam25.jpg}
  %               {http://www.pbrt.org/scenes_images/sanmiguel\_cam25.jpg}}
\caption{San Miguel Scene}
\label{fig:san_miguel}
\end{figure}

As an example we will consider the San Miguel data set, see 
Figure~\ref{fig:san_miguel}.  The scene was modeled by Guillermo M. Leal Llaguno
and is based on San Miguel de Allendo, Mexico.  It is a large scene with a 
nonuniform data distribution.  If we decompose the domain into twenty-seven 
spatially equal parts, we get the distribution shown in 
Figure~\ref{fig:san_miguel_data} a.  If we consider ray tracing this scene on a
machine with eight cores, we might get a distribution such as that shown
in Figure~\ref{fig:san_miguel_data} b.   This roughly distributes the data 
evenly between the nodes and positions neighbors on the same cores, giving us a
setup similar to what a pre-processing step designed to optimally distribute the 
data may compute.  Before we can conclude on which data decomposition pattern to 
use we need to consider the communication cost associated with each pattern.

\begin{figure}[!htb]
\minipage{0.52\textwidth}
  \includegraphics[height=3.2cm]{drawings/VoxelDistribution.pdf}
  
  (a) Triangles per Voxel
  
  \includegraphics[height=3.2cm]{drawings/DataDistribution.pdf}
  
  (c) Triangles per node  
\endminipage\hfill
\minipage{0.48\textwidth}
  \includegraphics[width=\linewidth]{drawings/NodeDistribution.pdf}  
  
  (b) Node distribution  
\endminipage
\caption{San Miguel}
\label{fig:san_miguel_data}
\end{figure}


\section{Communication} 
\label{sec:communication}
Each ray cast into a scene has the potential to interact with every triangle in
the scene due to reflection and refraction\footnote{ %
  Typically a maximum threshold is set to limit the number of times a ray
  can be reflected. 
}.  In addition each point being illuminated within a scene needs to cast 
secondary rays towards the lights.  These means determining which pieces of 
memory each ray will need throughout the ray tracing algorithm is not a straight
forward task.  With a distributed system and in a worst case scenario, each ray 
may need to communicate with every node.  

Communication is anticipated to be the bottleneck on an exascale system which
makes reducing communication cost high priority in our algorithm design.  
Optimally our goal is to create a ray tracing algorithm with distributed data 
that needs no communication.

\subsection{Communication Avoiding Ray Tracing}
\subsubsection{Ray Casting}
To design a communication avoiding ray tracer we will start with a simple ray
casting algorithm where only the primary viewing rays are traced. These rays are 
cast into a scene from the eye position.  If they intersect with an object, the 
ambient color is computed.  Without reflection, refraction and secondary rays,
communication between nodes executing the ray tracer can be avoided entirely.

\begin{figure}[!htb]
\minipage{0.4\textwidth}
\begin{algorithm}
TRACE_RAYS(voxels) 
  in: proces for each voxel of 
      data, sorted back to front
  out: image, a ray traced scene
  rays = COMPUTE_PRIMARY_RAYS
  for all voxel in voxels do
  rays_ = COPY(rays)
    voxel.TRACE_RAYS(rays_)
    for all ray in rays_ do
      if(ray.color) then
        image[ray.x][ray.y].color
         = ray.color;
      end if
    end for
  end for
return image
\end{algorithm}

(a) Controller code

\endminipage\hfill
\minipage{0.4\textwidth}
\begin{algorithm}
TRACE_RAYS(rays)
  in:  all primary rays
  out: all primary rays with 
       computed color
  for all ray in rays do
    if ray intersects scene then
      ray.color = COMPUTE_COLOR(ray)
    else
      ray.color = FALSE
    end if
  end for
return rays



.
\end{algorithm}

(b) Per voxel code

\endminipage\hfill
\caption{Ray casting pseudo code}
\label{fig:ray_caster}
\end{figure}

We can avoid all communication between nodes in a ray casting algorithm by 
preemptively sending every ray to every node, see Figure~\ref{fig:ray_caster} a.  
Each process executing over a chunk of data will receive every viewing ray, see 
Figure~\ref{fig:ray_caster} b.  Each process then traces the rays and computes a
color if the ray intersected an object within its data.  

\begin{figure}[!htb]
\minipage{0.5\textwidth}
  \includegraphics[width=\linewidth]{drawings/side.pdf}
  
(a) Side

\endminipage\hfill
\minipage{0.5\textwidth}
  \includegraphics[width=\linewidth]{drawings/front.pdf}
  
(b) Front

\endminipage\hfill
\caption{3D view of example cubes}
\label{fig:cubes_3d}
\end{figure}

The traced rays can then be used to produce the final image using a back to 
front ordering to ensure correct image composition.  As an illustration we can 
consider a simple scene with three cubes placed along a diagonal, 
see Figure~\ref{fig:cubes_3d}.  Each cube is broken into twelve triangles, two 
for each face.  If we distribute the data into eight spatially equal pieces and 
trace the scene from the camera view shown in Figure~\ref{fig:cubes_3d} b, we 
will get the eight images shown in Figure~\ref{fig:cubes}.  These eight images 
can then be composed back to front into the final image, 
Figure~\ref{fig:cubes_final}.

\begin{figure}[!htb]
\minipage{0.25\textwidth}
  \includegraphics[width=\linewidth]{drawings/cubes_01.pdf}
  Voxel 1 
  
  \includegraphics[width=\linewidth]{drawings/cubes_05.pdf}
  Voxel 5
\endminipage\hfill
\minipage{0.25\textwidth}
  \includegraphics[width=\linewidth]{drawings/cubes_02.pdf}
  Voxel 2
  
  \includegraphics[width=\linewidth]{drawings/cubes_06.pdf}
  Voxel 6
\endminipage\hfill
\minipage{0.25\textwidth}%
  \includegraphics[width=\linewidth]{drawings/cubes_03.pdf}
  Voxel 3
  
  \includegraphics[width=\linewidth]{drawings/cubes_07.pdf}
  Voxel 7
\endminipage
\minipage{0.25\textwidth}%
  \includegraphics[width=\linewidth]{drawings/cubes_04.pdf}
  Voxel 4
  
  \includegraphics[width=\linewidth]{drawings/cubes_08.pdf}
  Voxel 8
\endminipage
\caption{Images produced by each traced voxel}
\label{fig:cubes}
\end{figure}

\begin{figure}[!htb]
\centering
  \includegraphics[height=5cm]{drawings/cube_final.pdf}
\caption{Final composed image of traced voxels}
\label{fig:cubes_final}
\end{figure}


\subsubsection{Secondary light rays; light mesh}
Expanding on the ray casting algorithm in the previous section, we will now look
at how we might design similar communication avoiding algorithms to implement a 
full ray tracer.  We will first tackle secondary light rays, these are rays cast
from an intersection point to a light source.  To correctly compute shadows, it 
is important to know if the ray intersects with any other objects in the scene 
anywhere along its path from the intersection point of the object to the light 
source.  

Shadow ray calculations could result in a significant amount of communication if
done during runtime.  Since we are looking at static scenes, determining if a 
particular ray is in shadow can be pre-computed.  Similar to the ray casting 
algorithm, we can cast rays into the scene from each light source.  Instead of 
sending all rays to all voxels however, we will need to propagate the rays 
through the scene, starting with the light source and moving outward.  As the 
rays are propagated they can be marked as in shadow or not.  As the light rays 
reach a new voxel, they produce a mesh on the facing wall.  Each vertex in the 
mesh then would contain either an indicator that the vertex is in shadow or the 
direction and illumination information from its light source, 
see Figure~\ref{fig:light}.  Additional information on implementation can be 
found in Section~\ref{sec:proposed_algorithm}.

\begin{figure}[!htb]
\centering
\begin{subfigure}{0.49\textwidth}
 \centering
  \includegraphics[width=.98\columnwidth]{drawings/Lights1.pdf}
  \caption{Initial light rays}
\end{subfigure}
\begin{subfigure}{0.49\textwidth}
 \centering
  \includegraphics[width=.98\columnwidth]{drawings/Lights2.pdf}
  \caption{Computed light mesh}
\end{subfigure}
\caption{Light Ray Distribution}
\label{fig:light}
\end{figure}

This light mesh can then be used at runtime by the voxel computation to compute 
secondary light rays without needing to communicate with any other voxel.  The 
individual voxel, using the location of each light source can compute rays from 
an intersection point to each light source.  At the intersection of the ray to a
voxel wall, the corresponding light mesh will be used to determine the correct
illumination necessary for the intersection point.  A nearest neighbor or 
averaging algorithm can be used to determine which vertex in the mesh is closest
to the intersection point.  The information from the vertex can then be used for
the illumination calculation.

\subsubsection{Secondary reflected rays; material mesh}
A second set of secondary rays is needed to compute reflected rays.  A technique 
similar to the light meshes could be used, where a mesh is computed from each 
reflected material and distributed outward to each voxel.  Instead of holding 
light and shadow information the vertices of the mesh would contain material
information from the first object intersected.  Each voxel computation would 
then use the material mesh to determine the correct color for a reflected ray.
Each voxel would need access to all the material meshes in the case where a 
reflected ray reflects to another reflector.  See
Section~\ref{sec:proposed_algorithm} for implementation details.

If the scene contains little or no reflected rays, the overhead of computing the
material meshes may outweigh the cost of communicating reflected rays at 
runtime.  If the material meshes and the light meshes are used however 
communication cost would be reduced to one pass through the domain for each 
light and each reflector prior to runtime.  At runtime each voxel's computation
would be an independent calculation, requiring no communication except to send
back the final results of the traced rays.  The use of meshes however, results 
in an approximation of the result produced by a conventional ray tracer due to 
the nearest neighbor or averaging used when a ray intersects the mesh.

\subsubsection{Refracted rays}
Refracted materials which change the trajectory of the rays that pass through 
them are not handled by the proposed algorithm, see 
Section~\ref{sec:proposed_algorithm}.  The underlying assumptions to compute the
light and material meshes rely on the rays keeping a straight trajectory.  The
same assumption is made to compute the intersection of the primary viewing rays
and each voxel they intersect with.  If a ray refracts in one voxel, the 
subsequent voxels would need to know the new intersection point of that ray and 
their wall along with the rays new direction.  A potential solution to implement
refracted rays is presented in Section\~ref{sec:future-work}.

\section{Proposed Algorithm}
\label{sec:proposed_algorithm}
Reducing communication as described in Section~\ref{sec:communication} and using
the spatially even data distribution algorithm described in 
Section~\ref{sec:data_decomposition} allows us to design the communication 
avoiding ray tracer outlined in Figure~\ref{fig:design}.  The figure shows a 
Petri Net, see Section~\ref{sec:petri-nets}, describing the main components of 
the algorithm.  The following sections break down each place and transition.

\begin{figure}[!htb]
\centering
  \includegraphics[width=\linewidth]{drawings/Design.pdf}
\caption{Proposed Design Petri Net}
\label{fig:design}
\end{figure}

\subsection{Inputs}
The required inputs for the ray tracer are three places; lights, scene, and 
camera position.  There will be a single scene with potentially many lights.  
Additionally many camera positions can be traced, but each pass through the 
\emph{TRACED\_RAYS} will use a single camera position.  The labels on the arcs 
indicate the quantity relationships.

\begin{figure}[!htb]
\minipage{0.4\textwidth}
\begin{algorithm}
DISTRIBUTE_DATA(scene, v)
  in:  the scene to be traced
       number of voxels, v
  out: v subsets of data
  voxels[v]
  for all voxel in voxels do
    for all trangles in scene do
      if trangle is in voxel then
        voxel.data.add(trangle)
      end if
    end for
  end for
return voxels
\end{algorithm}
Distribute data
\endminipage\hfill
\caption{Ray tracer pseudo code}
\label{fig:ray_tracing_1}
\end{figure}

\subsection{Distribute data}
The transition, \emph{DISTRIBUTE\_DATA} takes a scene as an input and produces 
\emph{v} subsets of the data, one for each voxel.  \emph{Voxel Data} contins the
resulting data sets, see Figure~\ref{fig:ray_tracing_1}.

\subsection{Distribute rays}
The transition, \emph{DISTRIBUTE\_RAYS} takes a camera position as input and
produces a set of \emph{rays} viewing rays.  \emph{Primary Viewing Rays} are the
resulting computing sets.

\begin{figure}[!htb]
\minipage{0.4\textwidth}
\begin{algorithm}
TRACE_LIGHTS(lights, voxels) 
  in:  lights in the scene
       voxel data
  out: light mesh for each voxel
  for all light in lights do
    light_mesh = false;
    for all voxel in voxels do 
      if not light_mesh then
        light_mesh = 
          COMPUTE_MESH(light, voxel)
      else
        light_mesh = 
          PROPIGATE_MESH(light_mesh, 
                       light, voxel)
      end if
      light_mesh_ = COPY(light_mesh)
      voxel.light_meshes.add(light_mesh_)
    end for
  end for
return voxels


.
\end{algorithm}

(a) Trace lights

\endminipage\hfill
\minipage{0.4\textwidth}
\begin{algorithm}
TRACE_REFLECTORS(scene, voxels) 
  in:  scene to be traced
       voxel data
  out: material mesh for each reflector
  material_meshes[,]
  for all objects in scene do
    material_mesh = false
    if object is reflector then
      for all voxel in voxels do 
        if not light_mesh then
          material_mesh = 
            COMPUTE_MESH(object, voxel)
        else
          material_mesh = 
            PROPIGATE_MESH(material_mesh, 
                         object, voxel)
        end if
      end for
      material_meshes.add(object, 
                         material_mesh)
    end if
  end for
return material_meshes
\end{algorithm}

(b) Trace reflectors

\endminipage\hfill
\caption{Ray tracer pseudo code}
\label{fig:ray_tracing_2}
\end{figure}

\subsection{Trace lights}
The transition, \emph{TRACE\_LIGHTS} fires for each light in the domain and
uses the voxel data sets to produce a light mesh for each light for each
voxel.  \emph{Light Mesh} is the resulting mesh.  See 
Figure~\ref{fig:ray_tracing_2} a.

\subsection{Trace reflectors}
The transition, \emph{TRACE\_REFLECTORS} fires once for each scene and produces a
material mesh for every reflector in the scene for each voxel.  \emph{Reflector
Material Mesh} is the resulting mesh.  The pseudo code for trace reflectors is
similar to the pseudo code for trace lights, see Figure~\ref{fig:ray_tracing_2} 
b.

\subsection{Trace voxel}
The transition, \emph{TRACE\_VOXEL} fires once for each voxel and produces a 
copy of the viewing rays with computed color information.  \emph{Traced Rays} 
are the resulting rays.  This transition will use the light meshes created for 
its voxel along with the material meshes.  It will also need the primary viewing 
rays and the lights.  Pseudo code for trace voxel can be found in 
Figure~\ref{fig:ray_caster}.  Pseudo code for the helper method 
\emph{COMPUTE\_COLOR} is included in Section~\ref{sec:implementation}.


\begin{figure}[!htb]
\minipage{0.45\textwidth}
  \includegraphics[width=6cm]{drawings/Case_1.pdf}
  
  (a) Light rays with no obstruction, handled by light mesh
  
  \includegraphics[width=6cm]{drawings/Case_3.pdf}
  
  (c) Reflected rays, handled by material mesh
  
\endminipage\hfill
\minipage{0.45\textwidth}
  \includegraphics[width=6cm]{drawings/Case_2.pdf}
  
  (b) Light rays with obstruction, handled by light mesh
  
  \includegraphics[width=6cm]{drawings/Case_4.pdf}
  
  (d) Refracted rays, not supported
  
\endminipage
\caption{Ray tracing scene use cases}
\label{fig:use-cases}
\end{figure}


The algorithm presented here will handle ray casting as well as ray tracing for
scenes with multiple light sources and reflector material types.  Scenes with
refracted materials types are not supported.  The details on how the tracing
algorithm uses the light and material meshes can be found in 
Section~\ref{sec:implementation}.  A summary of the use cases handled and not 
handled can be seen in Figure\~ref{fig:use-cases}.

















 

\chapter{Implementation}
\label{chpt:implementation}

This chapter describes the implementation of the task-based 
communication-avoiding ray tracing algorithm designed in 
Chapter~\ref{chpt:design}. The system is implemented using the Intel 
implementation of CnC and is integrated with Embree.  Both libraries are 
introduced in Chapter~\ref{chpt:introduction} and described in
Chapter~\ref{chpt:previous-work}. We start with an overview of the CnC
specification file and then describe implementation details for the CnC tag,
data and step collections.

\begin{figure}[!htb]
  \centering
  \includegraphics[width=\textwidth]{drawings/CnC.pdf}
  \caption{CnC Graph}
  \label{fig:cnc}
\end{figure}

Figure~\ref{fig:cnc} shows a graphical subset of the full CnC specification (a
full textual version can be found in Figure~\ref{fig:cnc-graph-text}).  Step 
collections are represented as rectangles and data collections are represented
with ovals.  The tag collections are shown in the shape and surrounded with 
either square brackets or parenthesis depending on whether the tag collection is 
within a step collection or a data collection, respectively.  The colors are 
used to group the similar tag collections. The control collection is not 
represented in the graph but can be seen in the textual version. The graph 
begins execution when the object, luminaire, and camera information are
provided, and terminates when an image is produced.

\section{CnC Tag Collections}
\label{sec:tag-collections}
The elements of the tags in the tag collections used in the CnC specification
are \textbf{l} and \textbf{v}.  The \textbf{l} element is an index for a
specific luminaire.  The \textbf{v} element is the index of a specific voxel.
The tag collections, made up of sets of the tag elements, are 
\textbf{\{l\}}, \textbf{\{v\}}, and \textbf{\{l, v\}}.

\section{CnC Data Collections}
\label{sec:data-collections}
\begin{description}
\item[Scene] contains geometry and material information
for the scene being traced and comes from the environment.  In our 
implementation, this information is extracted from WaveFront
``.obj''~\cite{manual-wavefront} files.  A single scene is contained in the
collection for each image to trace.
\item[Camera] contains the location, direction, and field 
of view of the camera in addition to the resolution of the desired output image.
One entry for the camera is contained in the collection for each image to 
trace.
\item[Luminaire] contains information pertaining to each luminaire, including
the type of luminaire, location, direction when applicable, and the desired
number of generated light rays.  The current implementation supports point
and directional luminaires.  Each luminaire is indexed in the collection with a
unique identifier \emph{l}.
\item[Voxel] contains a subset of the geometry and 
material information from the \textbf{scene} data collection.  The subset is 
the information that pertains to the voxel it represents, indexed with the 
\emph{v} element of the collections tag.
\item[Shadow Mesh]contains a boolean value for each 
light ray associated with the indexed \emph{l}.  The collection contains an 
entry for each combination of \emph{l} and \emph{v}, 
\item[Traced Image] contains an array of illumination information and
intersection points for the indexed \emph{l}, \emph{v}.
\item[Image] contains the final composited image, the 
available formats of the output image are those supported by Embree. 
\end{description}

\section{CnC Step Collections}
\label{sec:implementation-step-collections}
\begin{description}
\item[Decompose Scene]reads a ``.obj'' file and produces 
subsets of the scene based on the voxel decomposition.  The number of voxels 
generated is determined by a command line argument and should be more than the 
number of nodes available.  Pseudocode can be found in 
Figure~\ref{fig:decompose-scene}.  The code implemented supports quad and 
triangle geometry as well as materials.  As Embree does not support 
textures, any material using a texture is modified to use a solid color.  
\item[Trace Light Rays] traces the light rays for a single luminaire 
and a single voxel.  Pseudocode for this step can be found in 
Figure~\ref{fig:trace-light-rays}.  Embree is used to determine if the light 
rays intersect any object within the voxel. The light rays are generated by
computing a uniform mesh along the bounding walls of the scene.  Each light ray
points towards a vertex on the mesh.  Light rays for point luminaires
originate from the origin of the luminaire.  Rays from directional luminaires do
not have an origin.  The number of light rays used is a parameter.
\item[Trace Viewing Rays] traces the voxel specified by the 
tag of the step instance using viewing rays generated by Embree based on the
camera information.  Pseudocode for this step can be found in
Figure~\ref{fig:trace-viewing-rays}.  Embree is used to determine if the
viewing rays intersect objects within the voxel.  The helper method referenced
in the pseudocode, \emph{illuminate}, is implemented using an ambient shading
model and a Blinn-Phong Shading model, respectively.  
\newpage
The shading models are implemented as follows:

$$ L^{ambient} = k^a L_a, $$
$$ L^{luminaire} = k^d ( \hat{s_i} \cdot \hat{n}) +
k^s ( \hat{s_i} \cdot \hat{h} )^{n_s} $$

\noindent where $L^{ambient}$ is the ambient radiance, $L^{luminaire}$ is the
luminaire's radiance, $k^a$ is the ambient reflectivity, $L_a$ is the ambient
irradiance, $k^d$ is the diffuse reflectivity, $\hat{s_i}$ is the direction
towards the luminaire, $\hat{n}$ is the surface normal, $k^s$ is the specular
reflectivity, $\hat{h}$ is the halfway vector and $n_s$ is the specular
exponent~\cite{cpts548}.

\item[Composite Image] produces a file containing the final ray traced image. 
Pseudocode can be found in Figure~\ref{fig:composite-image}.  The format of the
output image is set by a command line argument.

\end{description}












  
\chapter{Evaluation}
\label{sec:example}

Although purely theoretical, we can walk through a potential execution
of our algorithm and draw some conclusions on how it may perform.
Using the San Miguel data set
% RRL: citation?
and assuming we decompose the domain into 27 equal parts, we get the
distribution shown in Figure~\ref{fig:decomposition}. This will be the
cost of communicating the initial datasets to the nodes where
execution will take place. We will incur these costs again if data
needs to be moved once the algorithm starts executing.

\begin{figure}[!htb]
  \centering
  \includegraphics[width=0.5\textwidth]{drawings/DataDistribution.pdf}
  \caption{San Miguel Data Decomposition}
  \label{fig:decomposition}
\end{figure}

If we consider a machine with 8 cores, we might hope to get a distribution 
such as that shown in Figure~\ref{fig:machines}.   This roughly distributes
 the data evenly putting an emphasis on positioning neighbors on the same cores.  
 Depending on the lights in the scene, we also have to consider the cost of 
 distributing them, building the light mesh, and sending the light mesh data 
 to each node.  For this example I am assuming the costs associated with the 
 lighting is negligible next to the cost of distributed the data in the scene. 

>>> RESUME

\begin{figure}[!htb]
  \centering
  \includegraphics[width=0.5\textwidth]{drawings/NodeDistribution.pdf}
  \caption{Sample Data Distribution}
  \label{fig:machines}
\end{figure} 

Once the algorithm begins execution we can assume a worst case of 3 ray steps being necessary to trace rays from the camera through the scene.  This means each voxel will need to communicate with its neighbors a maximum of 3 times. Some of this communication will be within a node, some will be between nodes.  The cost of this communication needs to be factored into our estimates for total time to trace a given scene.  Acquiring accurate estimates for this equation will be the focus of our future work as exascale systems do not yet exist.  


\chapter{Conclusion and Future Work}
\label{future-work-and-conclusions}
\section{Conclusion}
\label{sec:conclusion}

We have taken the first steps in designing and implementing a ray tracing
algorithm for exascale.  Although we do not have exascale hardware yet, we have 
executed our implementation on today's systems, integrating the task-based 
library, CnC, with the ray tracing kernel library, Embree.  Our ray tracer does
not support all features of common ray tracers, we believe it provides a basis
that can be extended in future work.  We have shown the viability of a
communication-avoiding ray tracer and look forward to exploring aspects of
exascale such as hybrid systems to examine how they may impact our design.

\section{Future Work}
\label{sec:future-work}

This thesis presents a base system for task-based ray tracing that emphasizes 
communication reduction.  Several modifications to the algorithm could be made 
to improve runtime performance and feature support.  These improvements include
modifications to support GPU and CPU ray tracing, optimizing and future
evaluation on the implemented light meshes used, implementation of reflector
material meshes or another algorithm for reducing communication due to reflected
rays, and implementation of integrated restart functionality to support ray
refraction.  

\subsection{GPU ray tracing}
The exact landscape for future exascale distributed systems is unknown, but it 
has been suggested that the systems may have a combination of GPUs as well of 
CPUs to allow algorithms to take advantage of different hardware depending on 
their needs~\cite{kogge2013exascale}.  Our current ray tracing algorithm is 
integrated  with Embree, an optimized ray tracing library built for CPU�s.  
Embree could be swapped out for NVidia�s OptiX~\cite{parker2010optix}, a GPU 
based ray tracing engine.  Some scenes may show increased performance when ray 
traced on a GPU over a CPU and would benefit for the library swap.  In addition,
in the case of multiple viewpoints being rendered for a single static scene, 
there is the potential to detect and migrate data at runtime to the specific 
hardware a voxels data-set matches to improve runtime performance.  Exploring 
this realm of runtime optimizations is left for future work.

\subsection{Reflector material meshes}
Implementation and evaluation of the use of reflector meshes to reduce
communication is another area left for future work.  The concept introduced in 
section~\ref{sec:ca-ray-tracing} is similar to light meshes, introduced in the 
same section and detailed in section~\ref{sec:distribute-lights}.
Reflector material meshes could be used to eliminate the communication cost
associated with reflection, but come at a potentially large memory overhead
cost.  We will look at the details of creating and using a reflector mesh, then
evaluate the potential down-sides.

Pseudo code for creating a reflector mesh can be found in 
figure~\ref{fig:ray_tracing_2} b.  For a given reflective surface the first step 
would be to create a sample of points across the surface.  For each sampled 
point, a distribution of rays would be generated.  Each ray would originate from
the sample point and point in a different direction. The density of the surface
points and the distributed rays would need to be evaluated to determine optimal
sample sizes.  

Each of the distributed rays would then be cast from their origin into the
scene.  This casing could use a algorithm similar to the ray casting algorithm
presented in section ~\ref{sec:ca-ray-casting}. All distributed rays would be 
given to every voxel, if the ray intersects an object, material information from 
the intersected object would be stored, otherwise no information would be set.  
Once all voxels had computed the intersections, the results would be composed 
into a single data set we are calling a reflected material mesh.  This mesh 
would contain all the distribution rays for each sample point and each 
distribution ray would contain material information from the first object 
intersected.

The reflected material meshes would be passed to each voxel computation at
runtime to be used by the ray tracing algorithm.  When a ray intersected a
reflector, a quick lookup would give the appropriate mesh for the algorithm to
use.  A nearest neighbor approach could then be used to find the closest sample
point or points.  The direction of the reflected ray would need to be used to
determine which ray from the sample points distributed ray set should be used.
Once the ray is selected, the material information it contains could then be
used to compute the intersection color on the reflected material.

The direction of the reflected ray greatly affects the object that will be
reflected on a surface.  This results in the reflector material meshes differing
significantly from the light meshes.  Light meshes are attached to voxel walls
and samples with a single direction are sufficient to approximate illumination
calculations, see section~\ref{sec:trace-voxel}.  Reflector material meshes must
be sample and attached to every reflector surface.  In addition each point
requires potentially many rays to be evaluated.  This may significantly increase
the pre-processing cost as well as the memory overhead.  In a worst case
scenario a scene may have many reflectors all hidden behind a wall.  The
computation to compute the meshes would be done, but none of the viewing rays
would ever use the information, resulting in wasted memory and computation
space.  Optimizations and viability of this design as well as additional
communication reducing algorithms designed for handling reflected rays are left
to future work. 

\subsection{Refracted rays}
Extending our ray tracer to support refracted rays while keeping communication
costs low presents an opportunity to exploit another anticipated architecture
feature of exascale distributed systems.  Hardware failures are anticipated to
be more common at exascale~\cite{gropp2013programming}.  As a result, 
runtimes will need to detect failures and be able to recover from them.  One way
to do this is to periodically send memory information to a file and then use the
file to recover.  This capability could be leveraged by the application to 
manually invoke a failure when the system processes refraction.  The voxel 
computing incorrect data using non-refracted rays would be stopped.  The new 
trajectories of the refracted rays would be computed and the downstream voxel 
computations would be restarted with the new ray directions.


































\newpage
\singlespacing
\bibliographystyle{plain}
\bibliography{bibliography}

\appendix
% If you really *must* include source code.
\chapter{Source Code}
\begin{figure}[!htb]
  \begin{center}
    
\begin{algorithm}
/******************************************************************************
// * CnC Ray Tracer
// *
// * Author: Ellen Porter (ellen.porter@wsu.edu)
// *         Washington State University
// *
//****************************************************************************/

/******************************************************************************
// * Graph parameters */
$context {
  int voxels_i;   // decomposed domain size
  int voxels_j;
  int voxels_k;
  int num_frames; // frames to trace
};

/******************************************************************************
//* Item collection declarations */
//object data for each voxel, produced by decompose_domain
[ voxel_object *object_data : frame, i, j, k ];
// light data for each voxel, produced by distribute_lights
[ voxel_light *light_data : frame, i, j, k ];
// ray data passed to voxels, produced by distribute_rays
[ rays *primary_rays : frame ];
// ray data produced by trace_voxel
[ rays *traced_rays : frame, i, j, k ];

/******************************************************************************
//* CnC steps */
( decompose_domain : frame )
-> [ object_data : frame, 
     $range(0, #voxels_i), $range(0, #voxels_j), $range(0, #voxels_k) ];
( distribute_lights : frame )
<- [ object_data : frame, 
     $range(0, #voxels_i), $range(0, #voxels_j), $range(0, #voxels_k) ]
-> [ light_data : frame, 
    $range(0, #voxels_i), $range(0, #voxels_j), $range(0, #voxels_k) ];
( distribute_rays : frame )
-> [ primary_rays : frame ];
( trace_voxel : frame, i, j, k )
<- [ object_data  : frame, i, j, k],
   [ light_data   : frame, i, j, k],
   [ primary_rays : frame ]
-> [ traced_rays  : frame, i, j, k];
(compose_image : frame )
<- [ traced_rays : frame, 
     $range(0, #voxels_i), $range(0, #voxels_j), $range(0, #voxels_k)];

/******************************************************************************
//* Input output relationships from environment */
( $initialize: () )
-> (decompose_domain : $range(0, #num_frames)),
   (distribute_rays  : $range(0, #num_frames)),
   (trace_voxel      : $range(0, #num_frames), 
       $range(0, #voxels_i), $range(0, #voxels_j), $range(0, #voxels_k)),
   (compose_image    : $range(0, #num_frames));
( $finalize: () );

\end{algorithm} 
  \end{center}
  \caption{Textual CnC graph file}
  \label{fig:cnc-graph-text}
\end{figure}



\end{document}
