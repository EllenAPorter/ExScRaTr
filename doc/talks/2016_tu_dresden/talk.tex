\documentclass{beamer}
\usetheme{Dresden}

\usepackage{beamerthemesplit}
\usepackage{color}
\usepackage{fancyvrb}
\usepackage{framed}
\usepackage{graphicx}
\usepackage{listings}
\usepackage{verbatim}

\begin{comment}
% Insert these as needed.
  
% Any multidimensional value (point, vector, etc.)
\newcommand\Nd[1]{\mathbf{#1}}

% Arrays, Vectors, and Points are all bold, with additional (mostly
% LaTeX) conventions:
\newcommand\A[1]{\Nd{#1}}
% note: "\vec\Nd" does the right thing; "\Nd\vec" doesn't.
\newcommand\V[1]{\vec{\Nd{#1}}}
\newcommand\Pt[1]{\tilde{\Nd{#1}}}

% Normalize (as a function)
\newcommand\normalize[1]{\widehat{#1}}

% Unit-length vectors (with hats)
\newcommand\U[1]{\normalize{\Nd{#1}}}

% Components of unit-length vectors (with hats)
\newcommand\Usub[2]{\normalize{#1}_{#2}}

% Dot product
\newcommand\vdot[2]{({#1} \cdot {#2})}

% Flag forward references
\newcommand\tbd[1]{{\color{red} #1}}

% Rgb (multispectral) values
\newcommand\rgb[1]{\Nd{#1}}

% Matrices
\newcommand\mat[1]{\Nd{#1}}

% Vector magnitude
\newcommand\vmag[1]{\left|#1\right|}

% Absolute value of dot product
\newcommand\absvdot[2]{\left|{#1} \cdot {#2}\right|}

% Fourier transform
\newcommand{\Ft}[1]{\mathcal{F} \left[ #1 \right]}

% Fourier transform wrt a specified variable
\newcommand{\Ftwrt}[2]{\mathcal{F}_{#1} \left[ #2 \right]}

% sum over infinity wrt a specified variable
\newcommand{\suminftywrt}[1]{\sum_{#1=-\infty}^{\infty}}

% integrate over infinity
\newcommand{\intinfty}{\int_{-\infty}^{\infty}}

% make \sinc look like \sin
\newcommand{\sinc}{\mathsf{sinc}}
\end{comment}

\title{Exascale Raytracing}
\author{Bob Lewis}
\institute{School of Engineering and Applied Science \\
  Washington State University, Tri-Cities}
\date{March 15, 2016}

\begin{document}

\begin{frame}
  \titlepage
\end{frame}

\begin{comment}
% maybe
\begin{frame}
  \frametitle{Outline}
  \tableofcontents
\end{frame}
\end{comment}

\section{Exascale Computing}

\begin{frame}
  \frametitle{History}

  \begin{itemize}
  \item 1990's: parallel computing mainly a research area
  \item 1997: petascale ($10^{12}$ flops) achieved
  \item 2004: Moore's Law levels off -- improvements require parallel
    computation
  \item 2008: terascale ($10^{15}$ flops) achieved
  \item 201?: exascale $10^{18}$ flops achieved
  \end{itemize}
\end{frame}


\begin{frame}
  \frametitle{At the Exascale...}

  \begin{itemize}
  \item 100's - 1000's of cores per chip (die).
  \item Data movement time far exceeds computation time.
  \item Node failure will be a fact of life.
  \end{itemize}
\end{frame}


\begin{frame}
  \frametitle{Implications for Software}

  \begin{itemize}
  \item For the exascale, compilers need to know static data
    dependencies and locality.
  \item Recomputation may be cheaper than communication.
  \item There will be too many threads to manage explicity. (Analogy:
    register allocation)
  \item The runtime will need to be fault tolerant.
  \end{itemize}
\end{frame}


\begin{frame}
  \frametitle{Task-Based Programming Models}

  One modern approach to the exascale is ``task-based''.
  \begin{itemize}
  \item A static graph specifies connections between computation
    nodes and data locality.
  \item Inputs and outputs are explicit.
  \item The compiler will assign tasks to threads, possibly
    redundantly.
  \item The programmer will provide ``hints'' (aka pragmas) to the
    runtime.
  \item The execution order is determined by the runtime as inputs
    become ready.
  \end{itemize}
\end{frame}


\section{Parallel Ray Tracing}

\section{CnC}

% ---------------------------------------------------------------------------
\begin{comment}


\begin{frame}
  \frametitle{}

  \begin{itemize}
  \item 
  \end{itemize}
\end{frame}

\begin{frame}[fragile]
  \frametitle{}

  \begin{lstlisting}

  \end{lstlisting}
\end{frame}

\subsection{N: xxxx}

\begin{frame}
  \vfill
  \begin{center}
    {\huge Part N: xxxx}
  \end{center}
  \vfill
\end{frame}


\begin{frame}
  \frametitle{}

  \begin{columns}[c]
    \column{0.5\textwidth}
    \column{0.5\textwidth}
  \end{columns}
\end{frame}


\begin{frame}
  \frametitle{}

  \begin{center}
    \includegraphics[height=\textheight]{drawings/}
  \end{center}
\end{frame}


\end{comment}
% ---------------------------------------------------------------------------

\end{document}
